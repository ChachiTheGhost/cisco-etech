\subsection{IoT standards and protocols}
Several new protocols have been introduced specifically for IoT, some of which
are standardized:

\begin{enumerate}
  \item \textbf{RPL - IPv6 Routing Protocol for LLNs (RFC 6550):} RPL is a
  distance-vector routing protocol specifically designed to support IoT. At a
  high-level, RPL is a combination of control-plane and forwarding logic of
  three other technologies: regular IP routing, multi-topology routing (MTR),
  and MPLS traffic-engineering (MPLS TE). RPL is similar to regular IP routing
  in that directed acyclic graphs (DAG) are created through the network. This
  is a fancy way of saying ``loop-free shortest path'' between two points.
  These DAGs can be ``colored'' into different topologies which represent
  different network characteristics, such as high bandwidth or low latency.
  This forwarding paradigm is similar to MTR in concept. Last, traffic can be
  assigned to a colored DAG based on administratively-defined constraints,
  including node state, node energy, hop count, throughput, latency,
  reliability, and color (administrative preference). This is similar to MPLS
  TE’s constrained shortest path first (CSPF) process which is used for
  defining administrator-defined paths through a network based on a set of
  constraints, which might have technical and/or business drivers behind them.
  \item \textbf{6LoWPAN - IPv6 over Low Power WPANs (RFC 4944):} This
  technology was specifically developed to be an adaptation layer for IPv6 for
  IEEE 802.15.4 wireless networks. Specifically, it ``adapts'' IPv6 to work
  over LLNs which encompasses many functions:

  \begin{enumerate}
    \item \textbf{MTU correction:} The minimum MTU for IPv6 across a link, as
	defined in RFC2460, is 1280 bytes. The maximum MTU for IEEE 802.15.4 networks
	is 127 bytes. Clearly, no value can mathematically satisfy both conditions
	concurrently. 6LoWPAN performs fragmentation and reassembly by breaking the
	large IPv6 packets into IEEE 802.15.4 frames for transmission across the
	wireless network.
    \item \textbf{Header compression:} Many compression techniques are
	stateful and CPU-hungry. This strategy would not be appropriate for low-cost
	LLNs, so 6LoWPAN utilizes an algorithmic (stateless) mechanism to reduce the
	size of the IPv6 header and, if present, the UDP header. RFC4944 defines some
	common assumptions:

    \begin{enumerate}
      \item	The version is always IPv6.
      \item	Both source and destination addresses are link-local.
      \item	The low-order 64-bits of the link-local addresses can be derived
	  from the layer-2 network addressing in an IEEE 802.15.4 wireless network.
      \item	The packet length can be derived from the layer-2 header.
      \item	Next header is always ICMP, TCP, or UDP.
      \item	Flow label and traffic class are always zero.
      \item As an example, an IPv6 header (40 bytes) and a UDP header (8
	  bytes) are 48 bytes long when concatenated. This can be compressed down to
	  7 bytes by 6LoWPAN.
    \end{enumerate}

    \item \textbf{Mesh routing:} Somewhat similar to Wi-Fi, mesh networking is
	possible, but requires up to 4 unique addresses. The original
	source/destination addresses can be retained in a new ``mesh header'' while
	the per-hop source/destination addresses are written to the MAC header.
    \item \textbf{MAC level retransmissions:} IP was designed to be fully
	stateless and any retransmission or flow control was the responsibility of
	upper-layer protocols, such as TCP. When using 6LoWPAN, retransmissions can
	occur at layer-2.

  \end{enumerate}
  \item \textbf{CoAP - Constrained Application Protocol (RFC7252):} At a
  high-level, CoAP is very similar to HTTP in terms of the capabilities it
  provides. It is used to support the transfer of application data using
  common methods such as GET, POST, PUT, and DELETE. CoAP runs over UDP port
  5683 by default (5684 for secure CoAP) and was specifically designed to be
  lighter weight and faster than HTTP. Like the other IoT protocols, CoAP is
  designed for LLNs, and more specifically, to support machine-to-machine
  communications. Despite CoAP being designed for maximum efficiency, it is
  not a general replacement for HTTP. It only supports a subset of HTTP
  capabilities and should only be used within IoT environments. To interwork
  with HTTP, one can deploy an HTTP/CoAP proxy as a ``gateway'' device between
  the multi-service edge and smart device networks. CoAP has a number of useful
  features and characteristics:

  \begin{enumerate}
    \item \textbf{Supports multicast:} Because it is UDP-based, IP multicast
	is possible. This can be used both for application discovery (in lieu of DNS)
	or efficient data transfer.
    \item \textbf{Built-in security:} CoAP supports using datagram TLS (DTLS)
	with both pre-shared key and digital certificate support. As mentioned
	earlier, CoAP DTLS uses UDP port 5684.
    \item \textbf{Small header:} The CoAP overhead adds only 4 bytes.
    \item \textbf{Fast response:} When a client sends a CoAP GET to a server,
	the requested data is immediately returned in an ACK message, which is the
	fastest possible data exchange.
  \end{enumerate}

  \item \textbf{Message Queuing Telemetry Transport (ISO/IEC 20922:2016):}
  MQTT is, in a sense, the predecessor of CoAP in that it was created in 1999
  and was specifically designed for lightweight, web-based, machine-to-machine
  communications. Like HTTP, it relies on TCP, except uses ports 1883 and 8883
  for plain-text and TLS communications, respectively. Being based on TCP also
  implies a client/server model, similar to HTTP, but not necessary like CoAP.
  Compared to CoAP, MQTT is losing traction given the additional benefits
  specific to modern IoT networks that CoAP offers.
\end{enumerate}

The table that follows briefly compares CoAP, MQTT, and HTTP.

\begin{longtable}{llll}
\toprule
% top left cell is blank
&
\textbf{CoAP}
&
\textbf{MQTT}
&
\textbf{HTTP}
\\ \midrule
\textbf{Transport and ports}
&
UDP 5683/5684
&
TCP 1883/1889
&
TCP 80/443
\\ \midrule
\textbf{Security support}
&
DTLS via PSK/PKI
&
TLS via PSK/PKI
&
TLS via PSK/PKI
\\ \midrule
\textbf{Multicast support}
&
Yes, but no encryption support yet
&
No
&
No
\\ \midrule
\textbf{Lightweight}
&
Yes
&
Yes
&
No
\\ \midrule
\textbf{Standardized}
&
Yes
&
No; in progress
&
Yes
\\ \midrule
\textbf{Rich feature set}
&
No
&
No
&
Yes
\\
\bottomrule
\caption{IoT Transport Protocol Comparison}
\end{longtable}

Because IoT is so different than traditional networking, it is worth examining
some of the layer-1 and layer-2 protocols relevant to IoT. One common set of
physical layer enhancements that found traction in the IoT space are power
line communication (PLC) technologies. PLCs enable data communications
transfer over power lines and other electrical infrastructure devices. Two
examples of PLC standards are discussed below:

\begin{enumerate}
  \item \textbf{IEEE 1901.2-2013:} This specification allows for up to 500
  kbps of data transfer across alternating current, direct current, and

  non-energized electric power lines. Smart grid applications used to operate
  and maintain municipal electrical delivery systems can rely on the existing
  power line infrastructure for limited data communications.
  \item \textbf{HomePlug GreenPHY:} This technology is designed for home
  appliances such as refrigerators, stoves (aka ranges), microwaves, and even
  plug-in electric vehicles (PEV). The technology allows devices to be
  integrated with existing smart grid applications, similar to IEEE
  1901.2-2013 discussed above. The creator of this technology says that
  GreenPHY is a ``manufacturer's profile'' of the IEEE specification, and
  suggests that interworking is seamless.
\end{enumerate}

Ethernet has become ubiquitous in most networks. Originally designed for LAN
communications, it is spreading into the WAN via ``Carrier Ethernet'' and into
data center storage network via ``Fiber Channel over Ethernet'', to name a few
examples. In IoT, new ``Industrial Ethernet'' standards are challenging older
``field bus'' standards. The author describes some of the trade-offs between
these two technology sets below.

\begin{enumerate}
  \item	\textbf{Field bus:} Seen as a legacy family of technologies by some,
  field bus networks are still widely deployed in industrial environments.
  This is partially due to its incumbency and the fact that many endpoints on
  the network have interfaces that support various field bus protocols
  (MODBUS, CANBUS, etc). Field bus networks are economical as transmitting
  power over them is easier than power over Ethernet. Field bus technologies
  are less sensitive to electrical noise, have greater physical range without
  repeaters (copper Ethernet is limited to about 100 meters), and provide
  determinism to help keep machine communications synchronized.
  \item	\textbf{Industrial Ethernet:} To overcome the lack of deterministic
  and reliable transport of traditional Ethernet within the industrial sector,
  a variety of Ethernet-like protocols have been created. Two examples include
  EtherCAT and PROFINET. While speeds of industrial Ethernet are much slower
  than modern Ethernet (10 Mbps to 1 Gbps), these technologies introduce
  deterministic data transfer. In summary, these differences allow for
  accurate and timely delivery of traffic at slower speeds, compared to
  accurate and fast delivery at indeterminate times. Last, the physical
  connectors are typically ruggedized to further protect them in industrial
  environments.
\end{enumerate}

Standardization must also consider Government regulation. Controlling access
and identifying areas of responsibility can be challenging with IoT. Cisco
provides the following example:  For example, \textit{Smart Traffic Lights}
where there are several interested parties such as \textit{Emergency Services
(User), Municipality (owner), Manufacturer (Vendor)}. Who has provisioning
access? Who accepts Liability? \\

There is more than meets the eye with respect to standards and compliance for
street lights. Most municipalities (such as counties or townships within the
United States) have ordinances that dictate how street lighting works. The
light must be a certain color, must not ``trespass'' into adjacent streets,
must not negatively affect homeowners on that street, etc. This complicates
the question above because the lines become blurred between organizations
rather quickly. In cases like this, the discussions must occur between all
stakeholders, generally chaired by a Government/company representative
(depending on the consumer/customer), to draw clear boundaries between
responsibilities. \\

Radio frequency (RF) spectrum is a critical point as well. While Wi-Fi can
operate in the 2.4 GHz and 5.0 GHz bands without a license, there are no
unlicensed 4G LTE bands at the time of this writing. Deploying 4G LTE capable
devices on an existing carrier’s network within a developed country may not be
a problem. Deploying 4G LTE in developing or undeveloped countries, especially
if 4G LTE spectrum is tightly regulated but poorly accessible, can be a challenge.
