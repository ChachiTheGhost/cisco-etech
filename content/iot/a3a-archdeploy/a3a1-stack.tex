\subsection{IoT Technology Stack}
IoT, sometimes called Internet of Everything (IoE), is a concept that many
non-person entities (NPEs) or formerly non-networked devices in the world
would suddenly be networked. This typically includes things like window
blinds, light bulbs, water treatment plant sensors, home heating/cooling
units, street lights, and anything else that could be remotely controlled or
monitored. The business drivers for IoT are substantial: electrical devices
(like lights and heaters) could consume less energy by being smartly adjusted
based on changing conditions, window blinds can open and close based on the
luminosity of a room, and chemical levels can be adjusted in a water treatment
plant by networked sensors. These are all real-life applications of IoT and
network automation in general. \\

The term Low-power and Lossy Networks (LLN) is commonly used in the IoT space
since it describes the vast majority of IoT networks. LLNs have the following
basic characteristics (incomplete list):

\begin{enumerate}
  \item	Bandwidth constraints
  \item	Highly unreliable
  \item	Limited resources (power, CPU, and memory)
  \item	Extremely high scale (hundreds of millions and possibly more)
\end{enumerate}

Recently, the term Operational Technology (OT) has been introduced within the
context of IoT. OT is described by Gartner as hardware and software that
detects or causes a change through the direct monitoring and/or control of
physical devices, processes and events in the enterprise. OT encompasses the
technologies that organizations use to operate their businesses. \\

For example, a manufacturer has expensive machines, many of which use custom
protocols and software to operate. These machines have historically not been
tied into the Information Technology (IT) networking equipment that network
engineers typically manage. The concept of IT/OT convergence is made possible
by new developments in IoT. One benefit, as it pertains to the manufacturing
example, helps enhance ``just in time'' production. Market data from IT
systems tied into the Material Requirement Planning (MRP) system causes
production to occur only based on actual sales/demand, not a long-term
forecast. The result is a reduction in inventories (raw material, work in
process, and finished goods), lead time, and overall costs for a plant.

IoT combines a number of emerging technologies into its generalized network
architecture. The architecture consists primarily of four layers:

\begin{enumerate}
  \item	\textbf{Data center (DC) Cloud:} Although not a strict requirement,
  the understanding that a public cloud infrastructure exists to support IoT
  is a common one. A light bulb manufacturer could partner with a networking
  vendor to develop network-addressable light bulbs which are managed from a
  custom application running in the public cloud. This might be better than a
  private cloud solution since, if the application is distributed,
  regionalized instances could be deployed in geographically dispersed areas
  using an ``anycast'' design for scalability and performance improvements. As
  such, public cloud is generally assumed to be the DC presence for IoT networks.
  \item	\textbf{Core Networking and Services:} This could be a number of
  transports to connect the public cloud to the sensors. The same is true for
  any connection to public cloud, in reality, since even businesses need to
  consider the manner in which they connect to public cloud. The primary three
  options (private WAN, IXP, or Internet VPN) were discussed in the Cloud
  section. The same options apply here. A common set of technologies/services
  seen within this layer include IP, MPLS, mobile packet core, QoS, multicast,
  security, network services, hosted cloud applications, big data, and
  centralized device management (such as a network operations facility).
  \item	\textbf{Multi-service Edge (access network):} Like most SP networks,
  the access technologies tend to vary greatly based on geography, cost, and
  other factors. Access networks can be optically-based to provide Ethernet
  handoffs to IoT devices; this would make sense for relatively large devices
  that would have Ethernet ports and would be generally immobile. Mobile
  devices, or those that are small or remote, might use cellular technologies
  such as 2G, 3G, or 4G/LTE for wireless backhaul to the closest POP\@. A
  combination of the two could be used by extending Ethernet to a site and
  using 802.11 Wi-Fi to connect the sensors to the WLAN\@. The edge network may
  require use of ``gateways'' as a short-term solution for bridging
  (potentially non-IP) IoT networks into traditional IP networks. The gateways
  come with an associated high CAPEX and OPEX since they are custom devices to
  solve a very specific use-case. Specifically, gateways are designed to
  perform some subset of the following functions, according to Cisco:

  \begin{enumerate}
    \item \textbf{Map semantics between two heterogeneous domains:} The word
	semantics in this context refers to the way in which two separate networks
	operate and how each network interprets things. If the embedded systems
	network is a transparent radio mesh using a non-standard set of protocols
	while the multi-service edge uses IP over cellular, the gateway is
	responsible for ``presenting'' common interfaces to both networks. This
	allows devices in both networks to communicate using a ``language'' that is
	common to each.
    \item \textbf{Perform translation in terms of routing, QoS security,
	management, etc:} These items are some concrete examples of semantics. An
	appropriate analogy for IP networkers is stateless NAT64; an inside-local
	IPv4 host must send traffic to some outside-local IPv4 address which
	represents an outside-global IPv6 address. The source of that packet becomes
	an IPv6 inside-global address so that the IPv6 destination can properly reply.
    \item \textbf{Do more than just protocol changes:} The gateways serve as
	interworking devices between architectures at an architectural level. The
	gateways might have a mechanism for presenting network status/health between
	layers, and more importantly, be able to fulfill their architectural role in
	ensuring end-to-end connectivity across disparate network types.
  \end{enumerate}

  \item \textbf{Embedded Systems (Smart Things Network):} This layer
  represents the host devices themselves. They can be wired or wireless, smart
  or less smart, or any other classification that is useful to categorize an
  IoT component. Often times, such devices support zero-touch provisioning
  (ZTP) which helps with the initial deployment of massive-scale IoT
  deployments. For static components, these components are literally embedded
  in the infrastructure and should be introduced during the construction of a
  building, factory, hospital, etc. These networks are rather stochastic
  (meaning that behavior can be unpredictable). The author classifies wireless
  devices into three general categories which help explain what kind of
  RF-level transmission methods are most sensible:

  \begin{enumerate}
    \item \textbf{Long range:} Some devices may be placed very far from their RF base
	stations/access points and could potentially be highly mobile. Smart
	automobiles are a good example of this; such devices are often equipped with
	cellular radios, such as 4G/LTE\@. Such an option is not optimal for supporting
	LLNs given the cost of radios and power required to run them. To operate a
	private cellular network, the RF bands must be licensed (in the USA, at
	least), which creates an expensive and difficult barrier for entry.
    \item \textbf{Short range with ``better'' performance:} Devices that are within a
	local area, such as a building, floor of a large building, or courtyard area,
	could potentially use unlicensed frequency bands while transmitting at low
	power. These devices could be CCTV sensors, user devices (phones, tablets,
	laptops, etc), and other general-purpose things whereby maximum battery life
	and cost savings are eclipsed by the need for superior performance. IEEE
	802.11 Wi-Fi is commonly used in such environments. IEEE 802.16 WiMAX could
	also be used but, in the author’s experience, it is rare.
    \item \textbf{Short range with ``worse'' performance:} Many IoT devices
	fall into this final category whereby the device itself has a very small set
	of tasks it must perform, such as sending a small burst of data when an event
	occurs (i.e., some nondescript sensor). Devices are expected to be installed
	one time, rarely maintained, procured/operated at low cost, and be
	value-engineered to perform a limited number of functions. These devices are
	less commonly deployed in home environments since many homes have Wi-Fi; they
	are more commonly seen spread across cities. Examples might include street
	lights, sprinklers, and parking/ticketing meters. IEEE has defined 802.15.4
	to support low-rate wireless personal area networks (LR-PANS) which is used
	for many such IoT devices. Note that 802.15.4 is the foundation for
	upper-layer protocols such as ZigBee and WirelessHART\@. ZigBee, for example,
	is becoming popular in homes to network some IoT devices, such as
	thermostats, which may not support Wi-Fi in their hardware.
  \end{enumerate}
\end{enumerate}

IEEE 802.15.4 is worth a brief discussion by itself. Unlike Wi-Fi, all nodes
are ``full-function'' and can act as both hosts and routers; this is typical
for mesh technologies. A device called a PAN coordinator is analogous to a
Wi-Fi access point (WAP) which connects the PAN to the wired infrastructure;
this technically qualifies the PAN coordinator as a ``gateway'' discussed earlier.

\subsubsection{IoT Network Hierarchy}
The basic IoT architecture is depicted in the diagram that follows.

    \begin{minipage}[t]{\linewidth}
	  \centering
      \includegraphics[width=0.3\textwidth]{\imgpath iot-arch-alone.jpg}
      \captionof{figure}{IoT Network Architecture High Level}
    \end{minipage}

As a general comment, one IoT strategy is to ``mesh under'' and ``route
over''. This loosely follows the 7-layer OSI model by attempting to constrain
layers 1 and 2 to the IoT network, to include RF networking and link-layer
communications, then using some kind of IP overlay of sorts for network
reachability. Additional details about routing protocols for IoT are discussed
later in this document. \\

The mobility of an IoT device is going to be largely determined by its access
method. Devices that are on 802.11 Wi-Fi within a factory will likely have
mobility through the entire factory, or possibly the entire complex, but will
not be able to travel large geographic distances. For some specific
manufacturing work carts (containing tools, diagnostic measurement machines,
etc), this might be an appropriate method. Devices connected via 4G LTE will
have greater mobility but will likely represent something that isn’t supposed
to be constrained to the factory, such as a service truck or van. Heavy
machinery bolted to the factory floor might be wired since it is immobile. \\

Migrating to IoT need not be swift. For example, consider an organization
which is currently running a virtual private cloud infrastructure with some
critical in-house applications in their private cloud. All remaining
commercial applications are in the public cloud. Assume this public cloud is
hosted locally by an ISP and is connected via an MPLS L3VPN extranet into the
corporate VPN\@. If this corporation owns a large manufacturing company and
wants to begin deploying various IoT components, it can begin with the large
and immobile pieces. \\

The multi-service edge (access) network from the regional SP POP to the
factory likely already supports Ethernet as an access technology, so devices
can use that for connectivity. Over time, a corporate WLAN can be extended for
802.11 Wi-Fi capable devices. Assuming this organization is not deploying a
private 4G/5G LTE network, sensors can immediately be added using cellular as
well. Power line communication (PLC) technologies for transmitting data over
existing electrical infrastructure can also be used at this tier in the
architecture. The migration strategy towards IoT is very similar to adding new
remote branch sites, except the number of hosts could be very large. The LAN,
be it wired or wireless, still must be designed correctly to support all of
the devices. \\

Environment impacts are especially important for IoT given the scale of
devices deployed. Although wireless technologies become more resilient over
time, they remain susceptible to interference and other natural phenomena
which can degrade network connectivity. Some wireless technologies are even
impacted by rain, a common occurrence in many parts of the world. The
significance of this with IoT is to consider when to use wired or wireless
communications for a sensor. Some sensors may even be able to support multiple
communication styles in an active/standby design. As is true in most networks,
resilient design is important in ensuring that IoT-capable devices are operable. \\

\subsubsection{Data Acquisition and Flow}
Understanding data flow through an IoT network requires tracing all of the
communication steps from the sensors in the field up to the business
applications in the private data center or cloud. This is best explained with
an example to help solidify the high-level IoT architecture discussed in the
previous section. \\

Years ago, the author worked in a factory as a product quality assurance (QA)
technician for a large radio manufacturer. The example is based on a true
story with some events altered to better illustrate the relevance to IoT data
flow. \\

Stationed immediately after final assembly, newly-built products were tested
within customized test fixtures to ensure proper operation. The first series
of tests, for example, measured transmit power output, receiver sensitivity,
and other core radio functions. The next series required radios to be secured
to a large machine which would apply shock and vibration treatment to the
radios, ensuring they could tolerate the harsh treatment. The final series
consisted of environmental testing conducted in a temperature-controlled
chamber. The machine tested very hot temperatures, very cold temperatures, and
ambient temperature. Products had to pass all tests to be considered of
satisfactory quality for customer shipment. \\

None of this equipment was Ethernet or IP enabled, yet still had to report
test data back to a centralized system. This served a short term purpose of
tracking defects to redirect defective products to the rework area. It also
was useful in the long-term to identify trends relating to faulty product
design or testing procedures. All of this equipment is considered OT\@. \\

The test equipment described above is like an IoT sensor; it performs specific
measurements and reports the data back upstream. The first device in the path
is an IoT gateway, discussed in the previous section, which collects data from
many sensors. The gateway is responsible for reducing the data sent further
upstream. As such, a gateway is an aggregation node for many sensors. For
example, if 1 out of each 100 products fail QA, providing all relevant data
about the failed test is useful, but perhaps only a summary report about the
other 99 is sufficient. In this example, the IoT gateway was located in the
plant and connected into the corporate IT network. The gateway was, in effect,
an IT/OT interworking device. A Cisco IR 819 or IR 829 router is an example of
a gateway device. Additional intelligence (filtering, aggregation, processing,
etc.) could be added via fog/edge computing resources collocated with the IoT
gateway. \\

The gateway passes data to a data broker, such as a Cisco Kinetic Edge and Fog
Module (EFM) broker. This device facilitates communication between the IoT
network as a whole and the business applications. Thus, the broker is another
level of aggregation above the gateway as many gateways communicate to it. The
broker serves as an entry point (i.e., an API) for developers using Kinetic
EFM to tie into their IoT networks through the gateways. Please see the ``fog
computing'' section for more information on this solution, as it also includes
database functionality suited for IoT. \\

The data collected by the sensors can now be consumed by manufacturing
business applications, using the story described earlier. At its most benign,
the information presented to business leaders may just be used for reporting,
followed by human-effected downstream adjustments. For example, industrial
engineers may adjust the tolerance of a specific machine after noticing some
anomalies with the data. \\

A more modern approach to IoT would be using the business application's
intelligence to issue these commands automatically. Higher technology machines
can often times adjust themselves, and sometimes without human intervention.
The test equipment on the plant floor could be adjusted without the factory
employees needing to halt production to perform these tasks. The modifications
could be pushed from the business application down through the broker and
gateway to each individual device, but more commonly, an actor (opposite of a
sensor) is used for this purpose. The actor IoT devices do not monitor
anything about the environment, but serve to make adjustments to sensors based
on business needs. \\

The diagram that follows depicts a flow diagram of the IoT components and data flow
across the IoT architecture. For consistency, the same IoT high level diagram
is shown again but with additional details to support the aforementioned example.

    \begin{minipage}[t]{\linewidth}
	  \centering
      \includegraphics[width=0.6\textwidth]{\imgpath iot-arch-detail.jpg}
      \captionof{figure}{IoT Network Architecture With Example}
    \end{minipage}
