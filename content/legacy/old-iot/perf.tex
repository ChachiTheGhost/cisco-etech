\subsubsection{Performance, Reliability, and Scalability}
The performance of IoT devices is going to be a result of the desired security
and the access type. Many IoT devices will be equipped with relatively
inexpensive and weak hardware; this is sensible from a business perspective as
the device only needs to perform a few basic functions. This could be seen as
a compromise of security since strong ciphers typically require more
computational power for encryption/decryption functionality. In addition, some
IoT devices may be expected to last for decades while it is highly unlikely
that the same is true about cryptographic ciphers. In short, more expensive
hardware is going to be more secure and resilient. \\

The access type is mostly significant when performance is discussed. Although
4G LTE is very popular and widespread in the United States and other
countries, it is not available everywhere. Some parts of the world are still
heavily reliant on 2G/3G cellular service which is less capable and slower. A
widely distributed IoT network may have a combination of these access types
with various levels of performance. Higher performing 802.11 Wi-Fi speeds
typically require more expensive radio hardware, more electricity, and a
larger physical size. Physical access types (wired devices) will be generally
immobilized which could be considered a detriment to physical performance, if
mobility is required for an IoT device to do its job effectively.
