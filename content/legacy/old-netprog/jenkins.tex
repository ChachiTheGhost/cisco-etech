\subsubsection{Basic Jenkins Setup Demonstration}
Several CI/CD tools exist today. A common, open-source tool is known as
Jenkins which can be used for many CI/CD workflows. The feature list from
Jenkins’ website (included in the references) nicely summarizes the features
of the tool.

\begin{enumerate}
  \item \textbf{Continuous Integration and Continuous Delivery:} As an
  extensible automation server, Jenkins can be used as a simple CI server or
  turned into the continuous delivery hub for any project.
  \item \textbf{Easy installation:} Jenkins is a self-contained Java-based
  program, ready to run out-of-the-box, with packages for Windows, Mac OS X
  and other Unix-like operating systems.
  \item \textbf{Easy configuration:} Jenkins can be easily set up and
  configured via its web interface, which includes on-the-fly error checks and
  built-in help.  \item Plugins: With hundreds of plugins in the Update
  Center, Jenkins integrates with practically every tool in the continuous
  integration and continuous delivery toolchain.
  \item \textbf{Extensible:} Jenkins can be extended via its plugin
  architecture, providing nearly infinite possibilities for what Jenkins can do.
  \item \textbf{Distributed:} Jenkins can easily distribute work across
  multiple machines, helping drive builds, tests and deployments across
  multiple platforms faster.
\end{enumerate}

In this demonstration, the author explores two common Jenkins usages. The
first is utilizing the Git and Github plugins to create a ``build server'' for
code maintained in a repository. The demonstration will be limited to basic
Jenkins installation, configuration, and integration with a Github repository.
The actual testing of the code itself is a follow-on step that readers can
perform according to their CI/CD needs. This demonstration uses an Amazon
Linux EC2 instance in AWS, which is similar to Redhat Linux. \\

Before installing Jenkins on a target Linux machine, ensure Java 1.8.0 is
installed to prevent any issues. The commands below accomplish this, but the
outputs are not shown for brevity.

\begin{minted}{text}
yum install -y java-1.8.0-openjdk.x86_64
alternatives --set java /usr/lib/jvm/jre-1.8.0-openjdk.x86_64/bin/java
alternatives --set javac /usr/lib/jvm/jre-1.8.0-openjdk.x86_64/bin/javac
\end{minted}

To install Jenkins, issue these commands as root (indentation included for
readability). Technically, some of these commands can be issued from a
non-root user. The AWS installation guide for Jenkins, included in the
references, suggests doing so as root.

\begin{minted}{text}
wget -O /etc/yum.repos.d/jenkins.repo \
  http://pkg.jenkins-ci.org/redhat/jenkins.repo
rpm --import https://pkg.jenkins.io/redhat/jenkins.io.key
yum install jenkins
service jenkins start
\end{minted}

Verify Jenkins is working after the completing the installation. Also,
download the \verb|jenkins.war| file (~64MB) to get Jenkins CLI access, which is
useful for bypassing the GUI for some tasks. Because the file is large, users
may want to run it as a background task by appending \verb|&| to the command
(not shown). It is used below to check the Jenkins version.

\begin{minted}{text}
[root@ip-10-125-0-85 .ssh]# service jenkins status
jenkins (pid  2666) is running...

[root@ip-10-125-0-85 jenkins]# wget -q http://mirrors.jenkins.io/war/1.649/jenkins.war
[root@ip-10-125-0-85 jenkins]# java -jar jenkins.war --version
1.649
\end{minted}

Once Jenkins installed, log into Jenkins at http://jenkins.njrusmc.net:8080/,
substituting the correct hostname or IP address. Enable the relevant Git
plugins by navigating to \verb|Manage Jenkins > Manage Plugins > Available tab|.
Enter \verb|git| in the search bar. Select the plugs shown below and install
them. Each one will be installed, along with all their dependencies.

    \begin{minipage}[t]{\linewidth}
	  \centering
      \includegraphics[width=0.3\textwidth]{\imgpath jenkins-gitplugins.png}
      \captionof{figure}{Jenkins git Plugins}
    \end{minipage}

Once complete, log into Github and navigate to the user
\verb|Settings > Developer settings > Personal access tokens|. These tokens
can be used as an easy authentication method via shared-secret to access
Github's API\@. When generating a new token, \verb|admin:org_hook| must be granted at a
minimum, but in the interest of experimentation, the author selected a few
other options as depicted in the image that follows.

    \begin{minipage}[t]{\linewidth}
	  \centering
      \includegraphics[width=0.3\textwidth]{\imgpath jenkins-makegithubtoken.png}
      \captionof{figure}{Jenkins Personal Github Access Token}
    \end{minipage}

After the token has been generated and the secret text properly copied,
navigate to \verb|Credentials > Global Credentials| and create a new
credential. The graphic below depicts all parameters. This credential will be
used for the Github API integration.

    \begin{minipage}[t]{\linewidth}
	  \centering
      \includegraphics[width=0.3\textwidth]{\imgpath jenkins-tokencreds.png}
      \captionof{figure}{Jenkins Personal Access Tokens}
    \end{minipage}

Next, navigate to \verb|Manage Jenkins > Configure System|, then scroll down
to the Git and Github configurations. Configure the Git username and email
under the Git section. For the Github section, the secret text authentication
method should be used to allow Github API access.

    \begin{minipage}[t]{\linewidth}
	  \centering
      \includegraphics[width=0.3\textwidth]{\imgpath jenkins-gitpluginuser.png}
      \captionof{figure}{Jenkins User-specific Plugins}
    \end{minipage}

    \begin{minipage}[t]{\linewidth}
	  \centering
      \includegraphics[width=0.3\textwidth]{\imgpath jenkins-githubglobalconfig.png}
      \captionof{figure}{Jenkins Setting up Github Integration}
    \end{minipage}

The global Jenkins configuration for Git/Github integration is complete. Next,
create a new repository (or use an existing one) within Github. This process
is not described as Github basics are covered elsewhere in this book. The
author created a new repository called \verb|jenkins-demo|. \\

After creating the Github repository, the following commands are issued on the
user's machine to make a first commit. Github provides these commands in an
easy copy/paste format to get started quickly. The assumption is that the
user's laptop already has the correct SSH integration with Github.

\begin{minted}{text}
MacBook-Pro:jenkins-demo nicholasrusso$ echo "# jenkins-demo" >> README.md
MacBook-Pro:jenkins-demo nicholasrusso$ git init
Initialized empty Git repository in /Users/nicholasrusso/projects/jenkins-demo/.git/
MacBook-Pro:jenkins-demo nicholasrusso$ git add README.md
MacBook-Pro:jenkins-demo nicholasrusso$ git commit -m "first commit"
[master (root-commit) ac98dd9] first commit
 1 file changed, 1 insertion(+)
 create mode 100644 README.md
MacBook-Pro:jenkins-demo nicholasrusso$ git remote add origin \
>  git@github.com:nickrusso42518/jenkins-demo.git
MacBook-Pro:jenkins-demo nicholasrusso$ git push -u origin master

Counting objects: 3, done.
Writing objects: 100% (3/3), 228 bytes | 0 bytes/s, done.
Total 3 (delta 0), reused 0 (delta 0)
To git@github.com:nickrusso42518/jenkins-demo.git
 * [new branch]      master -> master
Branch master set up to track remote branch master from origin.
\end{minted}

After this initial commit, a simple Ansible playbook has been added as our
source code. Intermediate file creation and Git source code management (SCM)
steps are omitted for brevity, but there are now two commits in the Git log.
As it relates to Cisco Evolving Technologies, one would probably commit
customized code for Cisco Network Services Orchestration (NSO) or perhaps
Cisco-specific Ansible playbooks for testing. Jenkins would be able to access
these files, test it (or on a slave processing node within the Jenkins
system), and provide feedback about the build's quality. Jenkins can be
configured to initiate software builds (including compilation) using a variety
of tools and these builds can be triggered from a variety of events. These
features are not explored in detail in this demonstration.

\begin{minted}{yaml}
---
# sample-pb.yml
- hosts: localhost
  connection: local
  gather_facts: false
  tasks:
    - file:
        path: "/etc/ansible/ansible.cfg"
        state: present
...
\end{minted}

\begin{minted}{text}
MBP:jenkins-demo nicholasrusso$ git log --oneline --decorate
bb91945 (HEAD -> master, origin/master) Create sample-pb.yml
ac98dd9 first commit
\end{minted}

Next, log into the Jenkins box, wherever it exists (the author is using AWS
EC2 to host Jenkins for this demo on an \verb|m3.medium| instance). SSH keys
must be generated in the Jenkins users' home directory since this is the user
running the software. In the current release of Jenkins, the home directory is
\verb|/var/lib/jenkins/|.

\begin{minted}{text}
[root@ip-10-125-0-85 ~]# grep jenkins /etc/passwd
jenkins:x:498:497:Jenkins Automation Server:/var/lib/jenkins:/bin/false
\end{minted}

The intermediate Linux file system steps to create the \verb|~/.ssh/|
directory and \verb|~/.ssh/known_hosts| file are not shown for brevity.
Additionally, generating RSA2048 keys is not shown.  Navigating to the .ssh
directory is recommended since there are additional commands that use these files.

\begin{minted}{text}
[root@ip-10-125-0-85 .ssh]# pwd
/var/lib/jenkins/.ssh

[root@ip-10-125-0-85 .ssh]# ls
id_rsa  id_rsa.pub  known_hosts
\end{minted}

Next, add the Jenkins user's public key to Github under either your personal
username or a Jenkins utility user (preferred). The author uses his personal
username for brevity in this example shown in the diagram that follows.

\begin{minted}{text}
[root@ip-10-125-0-85 .ssh]# cat id_rsa.pub
ssh-rsa AAAAB3NzaC1yc2EAAAADAQABAAABAQDd6qISM3f/mhmSeauR6DSFMhvlT8QkXyyY73Tk8Nuf+SytelhP15gqTao3iA08LlpOBOnvtGXVwHEyQhMu0JTfFwRsTOGRRl3Yp9n6Y2/8AGGNTp+Q4tGpczZkh/Xs7LFyQAK3DIVBBnfF0eOiX20/dC5W72aF3IzZBIsNyc9Bcka8wmVb2gdYkj1nQg6VQI1C6yayLwyjFxEDgArGbWk0Z4GbWqgfJno5gLT844SvWmOWEJ1jNIw1ipoxSioVSSc/rsA0A3e9nWZ/HQGUbbhIOGx7k4ruQLTCPeduU+VgIIj3Iws1tFRwc+lXEn58qicJ6nFlIbAW1kJj8I/+1fEj jenkins-ssh-key
\end{minted}

    \begin{minipage}[t]{\linewidth}
	  \centering
      \includegraphics[width=0.3\textwidth]{\imgpath jenkins-gitsshkey.png}
      \captionof{figure}{Github SSH Keys for Jenkins Access}
    \end{minipage}

The commands below verify that the keys are functional. Note that the
\verb|-i| flag must be used because the command is run as root, and a
different identity file (Jenkins' user private key) should be used for this test.

\begin{minted}{text}
[root@ip-10-125-0-85 .ssh]# ssh -T git@github.com -i id_rsa
Hi nickrusso42518! You've successfully authenticated, but GitHub does not provide shell access.
\end{minted}

Before continuing, edit the \verb|/etc/passwd| file as root to give the
Jenkins user access to any valid shell (bash or sh). Additionally, use \verb|yum|
or \verb|apt-get| to install \verb|git| so that Jenkins can issue \verb|git|
commands. The \verb|git| installation via \verb|yum| is not shown for brevity.

\begin{minted}{text}
[root@ip-10-125-0-85 plugins]# grep jenkins /etc/passwd
jenkins:x:498:497:Jenkins Automation Server:/var/lib/jenkins:/bin/bash
\end{minted}

Once Git is installed and Jenkins has shell access, copy the repository URL in
SSH format from Github and substitute it as the repository argument in the
command below. This is the exact command that Jenkins tries to run when a
project's SCM is set to git; this tests reachability to a repository using
SSH\@. If you accidentally run the command as root, it will fail due to using
root's public key rather than Jenkins' public key. Switch to the Jenkins user,
try again, and test the return code (0 means success).

\begin{minted}{text}
[root@ip-10-125-0-85 plugins]# git ls-remote -h \
git@github.com:nickrusso42518/jenkins-demo.git HEAD
Permission denied (publickey).
fatal: Could not read from remote repository.

Please make sure you have the correct access rights
and the repository exists.

[root@ip-10-125-0-85 plugins]# su jenkins
bash-4.2$ git ls-remote -h \
>  git@github.com:nickrusso42518/jenkins-demo.git HEAD
bash-4.2$ echo $?
0
\end{minted}

The URL above can be copied by starting to clone the Github repository as
shown below. Be sure to select SSH to get the correct repository link.

    \begin{minipage}[t]{\linewidth}
	  \centering
      \includegraphics[width=0.3\textwidth]{\imgpath jenkins-repourl.png}
      \captionof{figure}{Github Repository URL for Jenkins Demo}
    \end{minipage}

At this point, adding a new Jenkins project should succeed when the repository
link is supplied. This is an option under SCM for the project whereby the only
choices are git and None. If it fails, an error message will be prominently
displayed on the screen and the error is normally related to SSH setup. Do not
specify any credentials for this because the SSH public key method is inherent
with the setup earlier. The screenshot that follows depicts this process.

    \begin{minipage}[t]{\linewidth}
	  \centering
      \includegraphics[width=0.3\textwidth]{\imgpath jenkins-projectscm.png}
      \captionof{figure}{Jenkins Source Code Management via git}
    \end{minipage}

As a final check, you can view the \verb|Console Output| for this
project/build by clicking the icon on the left. It reveals the git commands
executed by Jenkins behind the scenes to perform the pull, which is mostly
\verb|git fetch| to pull down new data from the Github repository associated
with the project.

\begin{minted}{text}
Started by user anonymous
Building in workspace /var/lib/jenkins/workspace/jenkins-demo
Cloning the remote Git repository
Cloning repository git@github.com:nickrusso42518/jenkins-demo.git
 > git init /var/lib/jenkins/workspace/jenkins-demo # timeout=10
Fetching upstream changes from git@github.com:nickrusso42518/jenkins-demo.git
 > git --version # timeout=10
 > git fetch --tags --progress git@github.com:nickrusso42518/jenkins-demo.git \
 >  +refs/heads/*:refs/remotes/origin/*
 > git config remote.origin.url git@github.com:nickrusso42518/jenkins-demo.git # timeout=10
[snip]
Commit message: "Create sample-pb.yml"
First time build. Skipping changelog.
Finished: SUCCESS
\end{minted}

The project workspace shows the files in the repository, which includes
the newly created Ansible playbook.

    \begin{minipage}[t]{\linewidth}
	  \centering
      \includegraphics[width=0.3\textwidth]{\imgpath jenkins-projectworkspace.png}
      \captionof{figure}{Jenkins }
    \end{minipage}

This section briefly explores configuring Jenkins integration with AWS EC2.
There are many more detailed guides on the Internet which describe this
process; this book includes the author's personal journey into setting it up.
Just like with Git, the AWS EC2 plugins must be installed. Look for the AWS
EC2 plugin as shown in the diagram that follows, and install it. The Jenkins wiki
concisely describes how this integration works and what problems it can solve:

\textit{
Allow Jenkins to start slaves on EC2 or Eucalyptus on demand, and kill them as
they get unused. With this plugin, if Jenkins notices that your build cluster
is overloaded, it'll start instances using the EC2 API and automatically
connect them as Jenkins slaves. When the load goes down, excessive EC2
instances will be terminated. This set up allows you to maintain a small
in-house cluster, then spill the spiky build/test loads into EC2 or another
EC2 compatible cloud.
}

    \begin{minipage}[t]{\linewidth}
	  \centering
      \includegraphics[width=0.3\textwidth]{\imgpath jenkins-ec2plugins.png}
      \captionof{figure}{AWS EC2 Plugin for Jenkins Integration}
    \end{minipage}

Log into the AWS console and navigate to the Identity Access Management (IAM)
service. Create a new user that has full EC2 access which effectively grants
API access to EC2 for Jenkins. The user will come with an access ID and secret
access key. Copy both pieces of information as Jenkins must know both.

    \begin{minipage}[t]{\linewidth}
	  \centering
      \includegraphics[width=0.3\textwidth]{\imgpath jenkins-iamuser.png}
      \captionof{figure}{Adding Jenkins user in AWS IAM}
    \end{minipage}

Next, create a new credential of type \verb|AWS credential|. Populate the
fields as shown below.

    \begin{minipage}[t]{\linewidth}
	  \centering
      \includegraphics[width=0.3\textwidth]{\imgpath jenkins-awscreds.png}
      \captionof{figure}{Jenkins AWS Credential Creation}
    \end{minipage}

Navigate back to \verb|Manage Jenkins > Configure System > Add a new cloud|.
Choose Amazon EC2 and populate the credentials option with the recently
created AWS credentials using the secret access key for the IAM user
\verb|jenkins|. You must select a specific AWS region. Additionally, you'll
need to paste the EC2 private key used for any EC2 instances managed by
Jenkins. This is not for general AWS API access but for shell access to EC2
instances in order to control them. For security, you can create a new key
pair within AWS (recommended but not shown) for Jenkins-based hosts in case
the general-purpose EC2 private key is stolen.

    \begin{minipage}[t]{\linewidth}
	  \centering
      \includegraphics[width=0.3\textwidth]{\imgpath jenkins-addcloud.png}
      \captionof{figure}{Adding AWS Cloud Option via Jenkins}
    \end{minipage}

You can validate the connection using the \verb|Test Connection| button which
should result in success.

    \begin{minipage}[t]{\linewidth}
	  \centering
      \includegraphics[width=0.3\textwidth]{\imgpath jenkins-awstestcon.png}
      \captionof{figure}{Testing Connection from AWS to Jenkins}
    \end{minipage}

The final step is determining what kind of AMIs Jenkins should create within
AWS\@. There can be multiple AMIs for different operating systems, including
Windows, depending on the kind of testing that needs to be done. Perhaps it is
useful to run the tests on different OS' as part of a more comprehensive
testing strategy for software portability. There are many options to enter and
the menu is somewhat similar to launching native instances within EC2. A
subset of options is shown here; note that you can validate the spelling of
the AMI codes (accessible from the AWS EC2 console) using the \verb|Check AMI|
button. More details on this process can be found in the references.

    \begin{minipage}[t]{\linewidth}
	  \centering
      \includegraphics[width=0.3\textwidth]{\imgpath jenkins-ami.png}
      \captionof{figure}{Jenkins AMIs within EC2}
    \end{minipage}

With both Github and AWS EC2 integration set up, a developer can create large
projects complete with automated testing from SCM repository and automatic
scaling within the public cloud. Provided there was a larger, complex project
which requires slave processing nodes, EC2 nodes would be dynamically created
based on the need or the administrator assigned labels within a project. \\

Jenkins is not the only commonly used CI/CD tool. Gitlab, which is private
(on-premises) version of Github, supports source code management (SCM) and
CI/CD together. A real-life example of this implementation is provided in the
references. All of these options come at a very low price and allow
individuals to deploy higher quality code more rapidly, which is a core tenant
of Agile software development. The author has participated in a number of free
podcasts on CI/CD and has used a variety of different providers. These podcasts
are linked in the references.
