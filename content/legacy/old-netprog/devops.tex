\subsubsection{DevOps methodologies, tools and workflows}
The term ``DevOps'' is relatively new and is meant to describe not just a job
title but a cultural shift in service delivery, management, and operation. It
was formerly known as ``agile system administration'' or ``agile
methodology''. The keyword ``agile'' typically refers to the integration of
development and operations staff throughout the entire lifecycle of a service.
The idea is to tear down the silos and the resulting poor service delivery
that both teams facilitate. Often times, developers will create applications
without understanding the constraints of the network, while the network team
will create a network (ineffective QoS, slow rerouting, etc) policies that
don’t support the business-critical applications. \\

The tools and workflows used within the DevOps community are things that
support an information sharing environment. Many of them are focused on
version control, service monitoring, configuration management, orchestration,
containerization, and everything else needed to typically support a service
through its lifecycle. The key to DevOps is that using a specific DevOps tool
does not mean an organization has embraced the DevOps culture or mentality. A
good phrase is ``People over Process over Tools'', as the importance of a
successful DevOps team is reliance on those things, in that order.

DevOps also introduces several new concepts. Two critical ones are continuous
integration (CI) and continuous delivery (CD). The CI/CD mindset suggests
several changes to traditional software development. Some of the key points
are listed here.

\begin{enumerate}
  \item	Everyone can see the changes: Dev, Ops, Quality Assurance (QA),
  management, etc.
  \item	Verification is an exact clone of the production environment, not
  simply a smoke-test on a developer’s test bed
  \item	The build and deployment/upgrade process is automated
  \item	Provide software in short timeframes and ensure releases are always
  available in increments
  \item	Reduce friction, increase velocity
  \item	Reduce silos, increase collaboration
\end{enumerate}

On the topic of software development models, it is beneficial to compare the
commonly used models with the new agile or DevOps mentality. Additional
details on these software development models can be found in the references. 
The table that follows contains a comparison chart of the different models.

\begin{longtable}{LLLL}
\toprule
% top left cell is blank
&
\textbf{Waterfall}
&
\textbf{Iterative}
&
\textbf{Agile}
\\ \midrule
\textbf{Basic Description}
&
Five serial phases, no feedback
\begin{enumerate}
  \item Requirements
  \item Design
  \item Implementation
  \item Verification
  \item Maintenance
\end{enumerate}
&
Like the waterfall model, but operates in loops. This creates a feedback
mechanism at each cycle to promote a faster and more flexible process.
&
Advances when the current step/function/feature is complete; cyclical model.
\\ \midrule
\textbf{Pros}
&
Simple, well understood, long history, requires minimal resources / management
oversight
&
Simple, well understood, opportunity to adjust, requires slightly more
resources than waterfall (but still reasonable)
&
Customer is engaged (better feedback), early detection of issues during rapid
code development periods (sprints)
\\ \midrule
\textbf{Cons}
&
Difficult to revert, customer is not engaged until the end, higher risk, slow
to deliver
&
Can be inefficient, customer feedback comes at the end of an iteration (not within)
&
High quantity of resources required, more focused management/customer
interaction needed
\\
\bottomrule
\caption{Software Development Methodology Comparison}
\end{longtable}

There are a number of popular Agile methodologies. Two of them are discussed below.

\begin{enumerate}
  \item \textbf{Scrum} is considered lightweight as the intent of most Agile
  methodologies is to maximize the amount of productive work accomplished
  during a given time period. In Scrum, a ``sprint'' is a period of time upon
  which certain tasks are expected to be accomplished. At the beginning of the
  sprint, the Scrum Master (effectively a task manager) holds a
  \textasciitilde{}4 hour planning meeting whereby work is prioritized and
  assigned to individuals.  Tasks are pulled from the sprint backlog into a
  given sprint. The only meetings thereafter (within a sprint) are typically
  15 minute daily stand-ups to report progress or problems (and advance items
  across the Scrum board). If a sprint is 2 weeks (~80 hours) then only about
  6 hours of it is spent in meetings. This may or may not include a
  retrospective discussion at the end of a sprint to discuss what went
  well/poorly. Tasks such as bugs, features, change requests, and more topics
  are tracked on a ``scrum board'' which drives the work for the entire sprint.
  \item \textbf{Kanban} is another Agile methodology which seeks to further
  optimize useful work done. Unlike scrum, it is less structured in terms of
  time and it lacks the concept of a sprint. As such, there is neither a
  sprint planning session nor a sprint retrospective. Rather than limit work
  by units of time, it limits work by the number of concurrent tasks occurring
  at once. The Kanban board, therefore, is more focused on tracking the number
  of tasks (sometimes called stories) within a single chronological point in
  the development cycle (often called Work In Progress or WIP). The most basic
  Kanban board might have three columns: To Do, Doing, Done. Ensuring that
  there is not too much work in any column keeps productivity high.
  Additionally, there is no official task manager in Kanban, though an
  individual may assume a similar role given the size/scope of the project.
  Finally, release cycles are not predetermined, which means releases can be
  more frequent.
\end{enumerate}

Although these Agile methodologies were initially intended for software
development, they can be adapted for work in any industry. The author has
personally seen Scrum used within a network security engineering team to
organize tasks, limit the scope of work over a period of time, and regularly
deliver production-ready designs, solutions, and consultancy to a large customer.
The author personally uses Kanban for personal task management, as well as
network operations and even home construction projects. Both strategies have
universal applicability.
