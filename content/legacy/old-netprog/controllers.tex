\subsubsection{SDN Controllers}
Controllers are components that are responsible for programming forwarding
tables of data-plane devices. Controllers themselves could even be routers,
like Cisco’s PfR operating as a master controller (MC), or they could be
software-only appliances, as seen with OpenFlow networks or Cisco’s
Application Policy Infrastructure Controller (APIC) used with ACI. The models
discussed above help detail the significance of the controller; this is
entirely dependent on the deployment model. The more involved a controller is,
the more flexibility the network administrator gains. This must be weighed
against the increased reliance on the controller itself. \\

A well-known example of an SDN controller is Open DayLight (ODL). ODL is
commonly used as the SDN controller for OpenFlow deployments. OpenFlow is the
communications protocol between ODL and the data-plane devices responsible for
forwarding packets (southbound). ODL communicates with business applications
via APIs so that the network can be programmed to meet the requirements of the
applications (northbound). \\

It is worth discussing a few of Cisco’s solutions in this section as they are
both popular with customers and relevant to Cisco’s vision of the future of
networking. Cisco’s Intelligent WAN (IWAN) is an evolutionary strategy to
bring policy abstraction to the WAN to meet modern design requirements, such
as path optimization, cost reduction via commodity transport, and transport
independence. IWAN has several key components:

\begin{enumerate}
  \item \textbf{Dynamic Multipoint Virtual Private Network (DMVPN):} This
  feature is a multipoint IP tunneling mechanism that allows sites to
  communicate to a central hub site without the hub needing to configure every
  remote spoke. Some variants of DMVPN allow for direct spoke-to-spoke traffic
  exchange using a reactive control-plane used to map overlay and underlay
  addresses into pairs. DMVPN can provide transport independence as it can be
  used as an overlay atop the public Internet, private WANs (MPLS), or any
  other transport that carries IP.
  \item \textbf{IP Service Level Agreement (IP SLA):} This feature is used to
  synthesize traffic to match application flows on the network. By sending
  probes that look like specific applications, IWAN can test application
  performance and make adjustments. This is called ``active monitoring''. The
  newest version of PfR (version 3) used within IWAN 2.0 no longer uses IP
  SLA. Instead, it uses Cisco-specific ``Smart Probes'' which provide some
  additional monitoring capabilities.
  \item \textbf{Netflow:} Like probes, Netflow is used to measure the
  performance of specific applications across an IWAN deployment, but does so
  without sending traffic. These measurements can be used to estimate
  bandwidth utilization, among other things. This is called ``passive monitoring''.
  \item \textbf{IP Routing:} Although not a new feature, some kind of overlay
  routing protocol is still needed. One of IWAN’s greatest strengths is that
  it can still rely on IP routing for a subset of flows, while choosing to
  optimize others. A total failure of the IWAN ``intelligence'' constructs
  will allow the WAN to fall back to classic IP routing, which is a known-good
  design and guaranteed to work. For this reason, existing design best
  practices and requirements gathering cannot be skipped when IWAN is deployed
  as these decisions can have significant business impacts.
  \item \textbf{Performance Routing (PfR):} PfR is the glue of IWAN that
  combines all of the aforementioned features into a comprehensive and
  functional system. It enhances IP routing in a number of ways:

  \begin{enumerate}
    \item Adjusting routing attributes, such as BGP local-preference, to
	prefer certain paths
    \item Injecting longer matches to prefer certain paths
    \item Installing dynamic route-maps for policy-routing when application
	packets are to be forwarded based on something other than their destination IP address
  \end{enumerate}

\end{enumerate}

When PfR is deployed, PfR speakers are classified as master controllers (MC)
or border routers (BR). MCs are the SDN ``controllers'' where policy is
configured and distributed. The BRs are relatively unintelligent in that they
consume commands from the MC and apply the proper policy. There can be a
hierarchy of MC/BR as well to provide greater availability for remote sites
that lose WAN connectivity. MCs are typically deployed in a stateless HA pair
using loopback addresses with variable lengths; the MCs typically exist in or
near the corporate data centers. \\

The diagram that follows depicts a high-level drawing of how IWAN works (from
Cisco’s IWAN wiki page). IWAN is generally positioned as an SD-WAN solution as
a way to connect branch offices to HQ locations, such as data centers.

    \begin{minipage}[t]{\linewidth}
	  \centering
      \includegraphics[width=0.6\textwidth]{\imgpath iwan-overview.png}
      \captionof{figure}{Cisco IWAN High Level Architecture}
    \end{minipage}
