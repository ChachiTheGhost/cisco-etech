% nfvi
\begin{longtable}{ll}
\toprule
\textbf{Advantage/Benefit}
&
\textbf{Disadvantage/Challenge}
\\ \midrule
Faster rollout of value added services
&
Likely to observe decreased performance
\\ \midrule
Reduced CAPEX and OPEX
&
Scalability exists only in purely NFV environment
\\ \midrule
Less reliance on vendor hardware refresh cycle
&
Interoperability between different VNFs
\\ \midrule
Mutually beneficial to SDN; complementary
&
Mgmt and orchestration alongside legacy systems
\\
\bottomrule
\end{longtable}



% cmtools
\begin{longtable}{lll}
\toprule
% top left cell is blank
&
\textbf{Git}
&
\textbf{Subversion (SVN)}
\\ \midrule
\textbf{General design}
&
Distributed; local and remote repo
&
Centralized; central repo only
\\ \midrule
\textbf{Staging area?}
&
Yes; can split work across commits
&
No, commit means push
\\ \midrule
\textbf{Learning curve}
&
Hard; many commands to learn
&
Easy; fewer moving pieces
\\ \midrule
\textbf{Branching and merging}
&
Easy, simple, and fast
&
Complex and laborious
\\ \midrule
\textbf{Revisions}
&
None; SHA1 commit IDs instead
&
Simple numbers; easy for non-techs
\\ \midrule
\textbf{Directory support}
&
Tracks only files, not directories
&
Tracks directories (empty ones too)
\\ \midrule
\textbf{Data tracked}
&
Content of the files
&
Files themselves
\\ \midrule
\textbf{Windows support}
&
Generally poor
&
Tortoise SVN plugin is a good option
\\
\bottomrule
\end{longtable}


% iot standards and protocols
\begin{longtable}{llll}
\toprule
% top left cell is blank
&
\textbf{CoAP}
&
\textbf{MQTT}
&
\textbf{HTTP}
\\ \midrule
\textbf{Transport and ports}
&
UDP 5683/5684
&
TCP 1883/1889
&
TCP 80/443
\\ \midrule
\textbf{Security support}
&
DTLS via PSK/PKI
&
TLS via PSK/PKI
&
TLS via PSK/PKI
\\ \midrule
\tetxbf{Multicast support}
&
Yes, but no encryption support yet
&
No
&
No
\\ \midrule
\textbf{Lightweight}
&
Yes
&
Yes
&
No
\\ \midrule
\textbf{Standardized}
&
Yes
&
No; in progress
&
Yes
\\ \midrule
\textbf{Rich feature set}
&
No
&
No
&
Yes
\\
\bottomrule
\end{longtable}


% controller based network design
\begin{longtable}{LLLLL}

\toprule
% top left cell is blank
&
\textbf{Distributed}
&
\textbf{Augmented}
&
\textbf{Hybrid}
&
\textbf{Centralized}
\\ \midrule
\textbf{Availability}
&
Dependent on the protocol convergence times and redundancy in the network
&
Dependent on the protocol convergence times and redundancy in the network.
Doesnt matter how bad the SDN controller is its failure is tolerable
&
Dependent on the protocol convergence times and redundancy in the network.
Doesnt matter how bad the SDN controller is  its failure is tolerable
&
Heavily reliant on a single SDN controller, unless one adds controllers to
split failure domains or to create resilience within a single failure domain
(which introduces a distributed control-plane in both cases)
\\ \midrule
\textbf{Granularity / control}
&
Generally low for IGPs but better for BGP. All devices generally need a common
view of the network to prevent loops independently. MPLS TE helps somewhat.
&
Better than distributed since policy injection can happen at the network edge,
or a small set of nodes. Can be combined with MPLS TE for more granular selection.
&
Moderately granular since SDN policy decisions are extended to all nodes. Can
influence decisions based on any arbitrary information within a datagram
&
Very highly granular; complete control over all routing decisions based on any
arbitrary information within a datagram
\\ \midrule
\textbf{Scalability}
&
Very high in a properly designed network (failure domain isolation, topology
summarization, reachability aggregation, etc)
&
High, but gets worse with more policy injection. Policies are generally
limited to key nodes (such as border routers)
&
Moderate, but gets worse with more policy injection. Policy is proliferated
across the network to all nodes (though the exact quantity may vary per node)
&
Depends; all devices retain state for all transiting flows. Hardware-dependent
on TCAM and whether SDN can use other tables such as L4 information, IPv6 flow
labels, etc
\\
\bottomrule
\end{longtable}




% controller based network design
\begin{longtable}{LLLLLL}
\toprule
\textbf{Method}
&
\textbf{Tx Target}
&
\textbf{Design}
&
\textbf{Operation}
&
\textbf{Used in}
&
\textbf{Network life}
\\ \midrule
\textbf{Direct Transmission}
&
BS
&
Point-to-point links back to BS
&
Send to BS independently
&
Homogenous
&
Poor
\\ \midrule
\textbf{LEACH}
&
CH
&
Proactive/ Cluster
&
Distributed (random)
&
Homogenous
&
Good (in general)
\\ \midrule
\textbf{LEACH-C}
&
CH
&
Proactive/ Cluster
&
Centralized assignment
&
Homogenous
&
Good (in general)
\\ \midrule
\textbf{TEEN}
&
CH
&
Reactive/ Cluster
&
Threshold-based alerts
&
Homogenous
&
Great (few comms)
\\ \midrule
\textbf{PEGASIS}
&
Neighbor
&
Greedy chain
&
Find closest node
&
Homogenous
&
Great (dense nodes)
\\ \midrule
\textbf{MTE}
&
Neighbor
&
Optimal chain
&
Find closest node
&
Homogenous
&
Great (sparse nodes)
\\ \midrule
\textbf{Static clustering}
&
CH
&
Cluster
&
Manual CH configuration
&
Heterogenous
&
Variable, but usually poor
\\ \midrule
\textbf{DEEC}
&
CH
&
Cluster
&
Distributed using initial energy only
&
Heterogenous
&
Good
\\ \midrule
\textbf{DDEEC/EDEEC}
&
CH
&
Cluster
&
Distributed using initial/residual energy
&
Heterogenous
&
Great
\\
\bottomrule
\end{longtable}

% legacy cloud comparison chart


\begin{longtable}{LLLLL}
\toprule
\textbf{Component/Utility}
\textbf{OpenStack}
\textbf{AWS}
\textbf{Microsoft Azure}
\textbf{GCP}
\textbf{Compute}
&
Nova
&
EC2
&
VMs
&
Compute Engine
\\ \midrule
\textbf{Network}
&
Neutron
&
VPC
&
Virtual Network
&
VPC
\\ \midrule
\textbf{Block Storage}
&
Cinder
&
EBS
&
Storage Disk
&
Persistent Disk
\\ \midrule
\textbf{Identity}
&
Keystone
&
IAM
&
Active Directory
&
Cloud IAM
\\ \midrule
\textbf{Image}
&
Glance
&
Lightsail
&
VMs and Images
&
Cloud Vision API
\\ \midrule
\textbf{Object Storage}
&
Swift
&
S3
&
Storage
&
Cloud Storage
\\ \midrule
\textbf{Dashboard}
&
Horizon
&
Console
&
Portal
&
Console
\\ \midrule
\textbf{Orchestration}
&
Heat
&
Batch
&
Batch
&
Cloud Dataflow
\\ \midrule
\textbf{Workflow}
&
Mistral
&
SWF
&
Logic Apps
&
Cloud Dataflow
\\ \midrule
\textbf{Telemetry}
&
Ceilometer
&
CloudWatch
&
VS App Insights
&
Cloud Pub/Sub
\\ \midrule
\textbf{Database}
&
Trove
&
RDS
&
SQL Database
&
Cloud Spanner
\\ \midrule
\textbf{Elastic Map Reduce}
&
Sahara
&
EMR
&
HDInsight
&
BigQuery and more
\\ \midrule
\textbf{Bare Metal}
&
Iconic
&
Dedicated instance
&
Bare Metal Recovery
&
Actual Cloud Platform
\\ \midrule
\textbf{Messaging}
&
Zaqar
&
SQS
&
Queue Storage
&
Cloud Pub/Sub
\\ \midrule
\textbf{Shared File System}
&
Manila
&
EFS
&
File Storage
&
FUSE
\\ \midrule
\textbf{DNS}
&
Designate
&
Route 53
&
DNS
&
Cloud DNS
\\ \midrule
\textbf{Search}
&
Searchlight
&
Elastic Search
&
Elastic Search
&
SearchService
\\ \midrule
\textbf{Key Manager}
&
Barbican
&
KMS
&
Key Vault
&
Cloud KMS
\bottomrule
\caption{Commercial Cloud Provider Comparison}
\end{longtable}





% legacy devops methodologies
\begin{longtable}{LLLL}
\toprule
% top left cell is blank
&
\textbf{Waterfall}
&
\textbf{Iterative}
&
\textbf{Agile}
\\ \midrule
\textbf{Basic Description}
&
Five serial phases, no feedback
\begin{enumerate}
  \item Requirements
  \item Design
  \item Implementation
  \item Verification
  \item Maintenance
\end{enumerate}
&
Like the waterfall model, but operates in loops. This creates a feedback
mechanism at each cycle to promote a faster and more flexible process.
&
Advances when the current step/function/feature is complete; cyclical model.
\\ \midrule
\textbf{Pros}
&
Simple, well understood, long history, requires minimal resources / management
oversight
&
Simple, well understood, opportunity to adjust, requires slightly more
resources than waterfall (but still reasonable)
&
Customer is engaged (better feedback), early detection of issues during rapid
code development periods (sprints)
\\ \midrule
\tetxbf{Cons}
&
Difficult to revert, customer is not engaged until the end, higher risk, slow
to deliver
&
Can be inefficient, customer feedback comes at the end of an iteration (not within)
&
High quantity of resources required, more focused management/customer
interaction needed
\\
\bottomrule
\caption{Software Development Methodology Comparison}
\end{longtable}


