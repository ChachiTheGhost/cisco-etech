\subsection{Infrastructure, platform, and software as a service (XaaS)}

Cisco defines four critical service layers of cloud computing:

\begin{enumerate}
  \item \textit{Software as a Service (SaaS) is where application services are
  delivered over the network on a subscription and on-demand basis.} A simple
  example would be to create a document but not installing the appropriate
  text editor on a user’s personal computer. Instead, the application is
  hosted ``as a service'' that a user can access anywhere, anytime, from any
  machine. SaaS is an interface between users and a hosted application, often
  times a hosted web application. Examples of SaaS include Cisco WebEx,
  Microsoft Office 365, github.com, blogger.com, and even Amazon Web Services
  (AWS) Lambda functions. This last example is particularly interesting since,
  according to Amazon, the \textit{``more granular model provides us with a
  much richer set of opportunities to align tenant activity with resource
  consumption''}. Being ``serverless'', lambda functions execute a specific task
  based on what the customer needs, and only the resources consumed during
  that task's execution (compute, storage, and network) are billed.

  \item \textit{Platform as a Service (PaaS) consists of run-time environments and
  software development frameworks and components delivered over the network on
  a pay-as-you-go basis. PaaS offerings are typically presented as API to
  consumers.} Similar to SaaS, PaaS is focused on providing a complete
  development environment for computer programmers to test new applications,
  typically in the development (dev) phase. Although less commonly used by
  organizations using mostly commercial-off-the-shelf (COTS) applications, it
  is a valuable offering for organizations developing and maintaining
  specific, in-house applications. PaaS is an interface between a hosted
  application and a development/scripting environment that supports it. Cisco
  provides WebEx Connect as a PaaS offering. Other examples of PaaS include
  the specific-purpose AWS services like Route 53 for Domain Name Service
  (DNS) support, CloudFront/CloudWatch for collecting performance metrics, and
  a wide variety of Relational Database Service (RDS) offerings for storing
  data. The customer consumes these services but does not have to maintain
  them (patching, updates, etc.) as part of their network operations.

  \item \textit{Infrastructure as a Service (IaaS) is where compute, network, and
  storage are delivered over the network on a pay-as-you-go basis. The
  approach that Cisco is taking is to enable service providers to move into
  this area.} This is likely the first thing that comes to mind when
  individuals think of ``cloud''. It represents the classic ``outsourced DC''
  mentality that has existed for years and gives the customer flexibility to
  deploy any applications they wish. Compared to SaaS, IaaS just provides the
  ``hardware'', roughly speaking, while SaaS provides both the underlying
  hardware and software application running on it. IaaS may also provide a
  virtualization layer by means of a hypervisor. A good example of an IaaS
  deployment could be a miniature public cloud environment within an SP point
  of presence (POP) which provides additional services for each customer:
  firewall, intrusion prevention, WAN acceleration, etc. IaaS is effectively
  an interface between an operating system and the underlying hardware
  resources. More general-purpose EC2 services such as Elastic Compute Cloud
  (EC2) and Simple Storage Service (S3) qualify as IaaS since the AWS'
  management is limited to the underlying infrastructure, not the objects
  within each service. The customer is responsible for basic maintenance
  (patching, hardening, etc.) of these virtual instances and data products.

  \item \textit{IT foundation is the basis of the above value chain layers. It
  provides basic building blocks to architect and enable the above layers.}
  While more abstract than the XaaS layers already discussed, the IT
  foundation is generally a collection of core technologies that evolve over
  time. For example, DC virtualization became very popular about 15 years ago
  and many organizations spent most of the last decade virtualizing ``as much
  as possible''. DC fabrics have also changed in recent years; the original
  designs represented a traditional core/distribution/access layer design yet
  the newer designs represent leaf/spine architectures. These are ``IT
  foundation'' changes that occur over time which help shape the XaaS
  offerings, which are always served using the architecture defined at this
  layer. Cisco views DC evolution in five phases:

  \begin{enumerate}
    \item \textbf{Consolidation:} Driven mostly by business needs to reduce costs,
	this phase focused on reducing edge computing and reducing the number of
	total DCs within an enterprise. DCs started to take form with two major
	components:

	\begin{enumerate}
      \item Data Center Network (DCN): Provides the underlying reachability
	  between attached devices in the DC, such as compute, storage, and
	  management tools.
      \item Storage Area Network (SAN): While this may be integrated or
	  entirely separate from the DCN, it is a core component in the DC\@. Storage
	  devices are interconnected over a SAN which typically extends to servers
	  needing to access the storage.
	\end{enumerate}

    \item \textbf{Abstraction:} To further reduce costs and maximize return on
	investment (ROI), this phase introduces pervasive virtualization. This
	provides virtual machine/workload mobility and availability to DC operators.

    \item \textbf{Automation:} To improve business agility, automation can
	rapidly and consistently ``do things'' within a DC\@. These things include
	routine system management, service provisioning, or business-specific tasks
	like processing credit card information.
    \item \textbf{Cloud:} With the previous phases complete, the cloud model of IT
	services delivered as a utility becomes possible for many enterprises. Such
	designs may include a mix of public and private cloud solutions.
    \item \textbf{Intercloud:} Discussed earlier, this is Cisco's vision of
	cloud interconnection to generally mirror the Internet concept. At this
	phase, internal and external clouds will coexist, federate, and share
	resources dynamically.
  \end{enumerate}
\end{enumerate}

Although not defined in formal Cisco documentation, there are many more
flavors of XaaS. Below are some additional examples of storage related
services commonly offered by large cloud providers:

\begin{enumerate}
  \item \textbf{Database-as-a-Service:} Some applications require databases,
  especially relational databases like the SQL family. This service would
  provide the database itself and the ability for the database to connect to
  the application so it can be utilized. AWS RDS services qualify as offerings
  in this category.
  \item \textbf{Object-Storage-as-a-Service:} Sometimes cloud users only need
  access to files independent from a specific application. Object storage is
  effectively a remote file share for this purpose, which in many cases can
  also be utilized by an application internally. This service provides the
  object storage service as well as the interfaces necessary for users
  and applications to access it. AWS S3 is an example of this service, which
  in some cases is a subset of IaaS/PaaS.
  \item \textbf{Block-Storage-as-a-Service:} These services are commonly tied to
  applications that require access to the disks themselves. Applications can
  format the disks and add whatever file system is necessary, or perhaps use
  the disk for some other purpose. This service provides the block storage
  assets (disks, logical unit number or LUNs, etc.) and the interfaces to
  connect the storage assets to the applications themselves. AWS Elastic Block
  Storage (EBS) is an example of this service.
\end{enumerate}

This book provides a more complete look into popular cloud service offerings
in the OpenStack section. Note OpenStack was removed from the new v1.1
blueprint but was retained at the end of this book.
