\subsection{Introduction}
Cisco has defined cloud as follows: \\

\textit{IT resources and services that are abstracted from the underlying infrastructure
and provided on-demand and at scale in a multitenant environment.}

Cisco identifies three key components from this definition that differentiate
cloud deployments from ordinary data center (DC) outsourcing strategies:

\textit{
\begin{enumerate}
  \item ``On-demand'' means that resources can be provisioned immediately when needed,
  released when no longer required, and billed only when used.
  \item ``At-scale'' means the service provides the illusion of infinite resource
  availability in order to meet whatever demands are made of it.
  \item ``Multitenant environment'' means that the resources are provided to many
  consumers from a single implementation, saving the provider significant costs.
\end{enumerate}
}

These distinctions are important for a few reasons. Some organizations joke that
migrating to cloud is simple; all they have to do is update their on-premises
DC diagram with the words ``Private Cloud'' and upper management will be
satisfied. While it is true that the term ``cloud'' is often abused, it is
important to differentiate it from a traditional private DC.
\\

Cloud architectures generally come in four variants:
\begin{enumerate}
  \item \textbf{Public:} Public clouds are generally the type of cloud most people think
  about when the word ``cloud'' is spoken. They rely on a third party organization
  (off-premise) to provide infrastructure where a customer pays a subscription
  fee for a given amount of compute/storage, time, data transferred, or any
  other metric that meaningfully represents the customer’s ``use'' of the cloud
  provider’s shared infrastructure. Naturally, the supported organizations do
  not need to maintain the cloud’s physical equipment. This is viewed by many
  businesses as a way to reduce capital expenses (CAPEX) since purchasing new
  DC equipment is unnecessary. It can also reduce operating expenses (OPEX)
  since the cost of maintaining an on-premise DC, along with trained staff,
  could be more expensive than a public cloud solution. A basic public cloud
  design is shown in the diagram that follows; the enterprise/campus edge uses some
  kind of transport to reach the Cloud Service Provider (CSP) network. The
  transport could be the public Internet, an Internet Exchange Point (IXP),
  a private Wide Area Network (WAN), or something else.

    \begin{minipage}[t]{\linewidth}
	  \centering
      \includegraphics[width=0.3\textwidth]{\imgpath cloud-basic-public.jpg}
      \captionof{figure}{Public Cloud High Level}
    \end{minipage}

% 
  % \begin{figure}[htbp!]
  % \centering
  % \includegraphics[width=0.3\textwidth]{\imgpath cloud-basic-public.jpg}
  % \caption{Public Cloud High Level}
  % \label{fig:public-cloud-high-level}
  % \end{figure}
% 
  \item \textbf{Private:} Like the joke above, this model is like an on-premises
  DC except it must supply the three key ingredients identified by Cisco to be
  considered a ``private cloud''. Specifically, this implies
  automation/orchestration, workload mobility, and compartmentalization must
  all be supported in an on-premises DC to qualify. The organization is
  responsible for maintaining the cloud’s physical equipment, which is
  extended to include the automation and provisioning systems. This can
  increase OPEX as it requires trained staff. Like the on-premises DC, private
  clouds provide application services to a given organization and
  multi-tenancy is generally limited to business units or projects/programs
  within that organization (as opposed to external customers). The diagram
  that follows illustrates a high-level example of a private cloud.

    \begin{minipage}[t]{\linewidth}
	  \centering
      \includegraphics[width=0.3\textwidth]{\imgpath cloud-basic-private.jpg}
      \captionof{figure}{Private Cloud High Level}
    \end{minipage}

  \item \textbf{Virtual Private:} A virtual private cloud is a combination of
  public and private clouds. An organization may decide to use this to offload
  some (but not all) of its DC resources into the public cloud, while
  retaining some things in-house. This can be seen as a phased migration to
  public cloud, or by some skeptics, as a non-committal trial. This allows a
  business to objectively assess whether the cloud is the ``right business
  decision''. This option is a bit complex as it may require moving workloads
  between public/private clouds on a regular basis. At the very minimum, there
  is the initial private-to-public migration; this could be time consuming,
  challenging, and expensive. This design is sometimes called a ``hybrid cloud''
  and could, in fact, represent a business’ IT end-state. The diagram that
  follows illustrates a high-level example of a virtual-private (hybrid) cloud.

    \begin{minipage}[t]{\linewidth}
	  \centering
      \includegraphics[width=0.3\textwidth]{\imgpath cloud-basic-vprivate.jpg}
      \captionof{figure}{Virtual Private Cloud High Level}
    \end{minipage}

  \item \textbf{Inter-cloud:} Like the Internet (an interconnection of various
  autonomous systems provide reachability between all attached networks),
  Cisco suggests that, in the future, the contiguity of cloud computing may
  extend between many third-party organizations. This is effectively how the
  Internet works; a customer signs a contract with a given service provider
  (SP) yet has access to resources from several thousand other service
  providers on the Internet. The same concept could be applied to cloud and
  this is an active area of research for Cisco.
\end{enumerate}

Below is a based-on-a-true-story discussion that highlights some of the
decisions and constraints relating to cloud deployments.

\begin{enumerate}
  \item An organization decides to retain their existing on-premises DC for
  legal/compliance reasons. By adding automation/orchestration and
  multi-tenancy components, they are able to quickly increase and decrease
  virtual capacity. Multiple business units or supported organizations are
  free to adjust their security policy requirements within the shared DC in a
  manner that is secure and invisible to other tenants; this is the result of
  compartmentalization within the cloud architecture. This deployment would
  qualify as a ``private cloud''.

  \item Years later, the same organization decides to keep their most
  important data on-premises to meet seemingly-inflexible Government
  regulatory requirements, yet feels that migrating a portion of their private
  cloud to the public cloud is a solution to reduce OPEX long term. This increases
  the scalability of the systems for which the Government does not regulate,
  such as virtualized network components or identity services, as the
  on-premises DC is bound by CAPEX reductions. The private cloud footprint can
  now be reduced as it is used only for a subset of tightly controlled
  systems, while the more generic platforms can be hosted from a cloud
  provider at lower cost. Note that actually exchanging/migrating workloads
  between the two clouds at will is not appropriate for this organization as
  they are simply trying to outsource capacity to reduce cost. As discussed
  earlier, this deployment could be considered a ``virtual private cloud'' by
  Cisco, but is also commonly referred to as a ``hybrid cloud''.

  \item Years later still, this organization considers a full migration to the
  public cloud. Perhaps this is made possible by the relaxation of the
  existing Government regulations or by the new security enhancements offered
  by cloud providers. In either case, the organization can migrate its
  customized systems to the public cloud and consider a complete decommission
  of their existing private cloud. Such decommissioning could be done
  gracefully, perhaps by first shutting down the entire private cloud and
  leaving it in ``cold standby'' before removing the physical racks. Rather than
  using the public cloud to augment the private cloud (like a virtual private
  cloud), the organization could migrate to a fully public cloud solution.
\end{enumerate}

Cloud implementation can be broken into 2 main categories: how the cloud
provider works, and how customers connect to the cloud. The second question is
more straightforward to answer and is discussed first. There are three main
options for connecting to a cloud provider, but this list is by no means
exhaustive:

\begin{enumerate}
  \item \textbf{Private WAN (like MPLS L3VPN):} Using the existing private WAN, the
  cloud provider is connected as an extranet. To use MPLS L3VPN as an example,
  the cloud-facing PE exports a central service route-target (RT) and imports
  corporate VPN RT. This approach could give direct cloud access to all sites
  in a highly scalable, highly performing fashion. Traffic performance would
  (should) be protected under the ISP’s SLA to cover both site-to-site
  customer traffic and site-to-cloud/cloud-to-site customer traffic. The ISP
  may even offer this cloud service natively as part of the service contract.
  Certain services could be collocated in an SP POP as part of that SP's cloud
  offering. The private WAN approach is likely to be expensive and as
  companies try to drive OPEX down, a private WAN may not even exist. Private
  WAN is also good for virtual private (hybrid) cloud assuming the ISP’s SLA
  is honored and is routinely measuring better performance than alternative
  connectivity options. Virtual private cloud makes sense over private WAN
  because the SLA is assumed to be better, therefore the intra-DC traffic
  (despite being inter-site) will not suffer performance degradation. Services
  could be spread between the private and public clouds assuming the private
  WAN bandwidth is very high and latency is very low, both of which would be
  required in a cloud environment. It is not recommended to do this as the
  amount of intra-workflow bandwidth (database server on-premises and
  application/web server in the cloud, for example) is expected to be very
  high. The diagram that follows depicts private WAN connectivity
  assuming MPLS L3VPN. In this design, branches could directly access cloud
  resources without transiting the main site.

    \begin{minipage}[t]{\linewidth}
	  \centering
      \includegraphics[width=0.7\textwidth]{\imgpath cloud-private-wan.jpg}
      \captionof{figure}{Connecting Cloud via Private WAN}
    \end{minipage}

  \item \textbf{Internet Exchange Point (IXP):} A customer’s network is
  connected via the IXP LAN (might be a LAN/VLAN segment or a layer-2 overlay)
  into the cloud provider’s network. The IXP network is generally access-like
  and connects different organizations together so that they can peer with
  Border Gateway Protocol (BGP) directly, but typically does not provide
  transit services between sites like a private WAN. Some describe an IXP as a
  ``bandwidth bazaar'' or ``bandwidth marketplace'' where such exchanges can
  happen in a local area. A strict SLA may not be guaranteed but performance
  would be expected to be better than the Internet VPN. This is likewise an
  acceptable choice for virtual private (hybrid) cloud but lacks the tight SLA
  typically offered in private WAN deployments. A company could, for example,
  use internet VPNs for inter-site traffic and an IXP for public cloud access.
  A private WAN for inter-site access is also acceptable.

    \begin{minipage}[t]{\linewidth}
	  \centering
      \includegraphics[width=0.7\textwidth]{\imgpath cloud-ixp.jpg}
      \captionof{figure}{Connecting Cloud via IXP}
    \end{minipage}

  \item \textbf{Internet VPN:} By far the most common deployment, a customer
  creates a secure VPN over the Internet (could be multipoint if outstations
  require direct access as well) to the cloud provider. It is simple and cost
  effective, both from a WAN perspective and DC perspective, but offers no SLA
  whatsoever. Although suitable for most customers, it is likely to be the
  most inconsistently performing option. While broadband Internet connectivity
  is much cheaper than private WAN bandwidth (in terms of price per Mbps), the
  quality is often lower. Whether this is ``better'' is debatable and depends on
  the business drivers. Also note that Internet VPNs, even high bandwidth
  ones, offer no latency guarantees at all. This option is best for fully
  public cloud solutions since the majority of traffic transiting this VPN
  tunnel should be user service flows. The solution is likely to be a poor
  choice for virtual private clouds, especially if workloads are distributed
  between the private and public clouds. The biggest drawback of the Internet
  VPN access design is that slow cloud performance as a result of the
  ``Internet'' is something a company cannot influence; buying more bandwidth is
  the only feasible solution. In this example, the branches don’t have direct
  Internet access (but they could), so they rely on an existing private WAN to
  reach the cloud service provider.

    \begin{minipage}[t]{\linewidth}
	  \centering
      \includegraphics[width=0.7\textwidth]{\imgpath cloud-internet-vpn.jpg}
      \captionof{figure}{Connecting Cloud via Internet VPN}
    \end{minipage}
\end{enumerate}

The answer to the first question detailing how a cloud provider network is
built, operated, and maintained is discussed in the remaining sections.
