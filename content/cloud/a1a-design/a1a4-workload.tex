\subsection{Workload migration}
Workload mobility is a generic goal and has been around since the first
virtualized DCs were created. This gives IT administrators an increased
ability to share resources amount different workloads within the virtual DC
(which could consist of multiple DCs connected across a Data Center
Interconnect, or DCI). It also allows workloads to be balanced across a
collection of resources. For example, if 4 hosts exist in a cluster, one of
them might be performing more than 50\% of the computationally-expensive work
while the others are underutilized. The ability to move these workloads is an
important capability. \\

It is important to understand that workload mobility is not necessarily the
same thing as VM mobility. For example, a workload’s accessibility can be
abstracted using anycast while the application exists in multiple availability
zones (AZ) spread throughout the cloud provider’s network. Using Domain Name
System (DNS), different application instances can be utilized based on
geographic location, time of day, etc. The VMs have not actually moved but the
resource performing the workload may vary. \\

Although this concept has been around since the initial virtualization
deployments, it is even more relevant in cloud, since the massively scalable
and potentially distributed nature of that environment is abstracted into a
single ``cloud'' entity. Using the cluster example from above, those 4 hosts
might not even be in the same DC, or even within the same cloud provider (with
hybrid or Inter-cloud deployments). The concept of workload mobility needs to
be extended large-scale; note that this doesn’t necessarily imply layer-2
extensions across the globe. It simply implies that the workload needs to be
moved or distributed differently, which can be solved with
geographically-based anycast solutions, for example. \\

As discussed in the automation/orchestration section above, orchestrating
workloads is a major goal of cloud computing. The individual tasks that are
executed in sequence (and conditionally) by the orchestration engine could be
distributed throughout the cloud. The task itself (and the code for it) is
likely centralized in a code repository, which helps promote the
``infrastructure as code'' concept. The task/script code can be modified,
ultimately changing the infrastructure without logging into individual
devices. This has CM benefits for the managed device, since the device's
configuration does not need to be under CM at all anymore.
