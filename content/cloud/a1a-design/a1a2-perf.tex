\subsection{Performance, scalability, and high availability}
Assessing the performance and reliability of cloud networks presents an
interesting set of trade-offs. For years, network designers have considered
creating ``failure domains'' in the network so as to isolate faults. With
routing protocols, this is conceptually easy to understand, but often times
difficult to design and implement, especially when considering
business/technical constraints. Designing a DC comes with its own set of
trade-offs when identifying the ``failure domains'' (which are sometimes called
``availability zones'' within a fabric), but that is outside the scope of this
document. The real trade-offs with a cloud environment revolve around the
introduction of automation. Automation is discussed in detail elsewhere,
but the trade-offs are discussed here as they directly influence the
performance and reliability of a system. Note that this discussion is
typically relevant for private and virtual private clouds, as a public cloud
provider will always be large enough to warrant several automation tools. \\

Automation usually reduces the total cost of ownership (TCO), which is
desirable for any business. This is the result of reducing the time (and labor
wages) it takes for individuals to ``do things'': provision a new service,
create a backup, add VLANs to switches, test MPLS traffic-engineering tunnel
computations, etc. The trade-off is that all software (including the
automation system being discussed) requires maintenance, whether that is in
the form of in-house development or a subscription fee from a third-party. If
in the form of in-house development, software engineers are paid to maintain
and troubleshoot the software which could potentially be more expensive than
just doing things manually, depending on how much maintenance and unit testing
the software requires. Most individuals who have worked as software developers
(including the author) know that bugs or feature requests always seem to pop
up, and maintenance is continuous for any non-trivial piece of code.
Businesses must also consider the cost of the subscription for the automation
software against the cost of not having it (in labor wages). Typically this
becomes a simple choice as the network grows; automation often shines here.
Automation is such a key component of cloud environments because the cost of
dealing with software maintenance is almost always less than the cost of a
large IT staff. \\

Automation can also be used for root cause analysis (RCA) whereby the tool can
examine all the components of a system to test for faults. For example,
suppose an eBGP session fails between two organizations. The script might test
for IP reachability between the eBGP routers first, followed by verifying no
changes to the infrastructure access lists applied on the interface. It might
also collect performance characteristics of the inter-AS link to check for
packet loss. Last, it might check for fragmentation on the link by sending
large pings with ``don’t fragment'' set. This information can feed into the RCA
which is reviewed by the network staff and presented to management after an
outage. \\

The main takeaway is that automation should be deployed where it makes sense
(TCO reduction) and where it can be maintained with a reasonable amount of
effort. Failing to provide the maintenance resources needed to sustain an
automation infrastructure can lead to disastrous results. With automation, the
``blast radius'', or potential scope of damage, can be very large. A real-life
story from the author: when updating SNMPv3 credentials, the wrong privacy
algorithm was configured, causing 100\% of devices to be unmanageable via
SNMPv3 for a short time. Correcting the change was easily done using
automation, and the business impact was minimal, but it negatively affected
every router, switch, and firewall in the network. \\

Automation helps maximize the performance and reliability of a cloud
environment. Another key aspect of cloud design is accessibility, which
assumes sufficient network bandwidth to reach the cloud environment. A DC that
was once located at a corporate site with 2,000 employees was accessible to
those employees over a company’s campus LAN architecture. Often times this
included high-speed core and DC edge layers whereby accessing DC resources was
fast and highly available. With public cloud, the Internet/private WAN becomes
involved, so cloud access becomes an important consideration. \\

Achieving cloud scalability is often reliant on many components supporting the
cloud architecture. These components include the network fabric, the
application design, the virtualization/segmentation design, and others. The
ability of cloud networks to provide seamless and simple interoperability
between applications can be difficult to assess. Applications that are written
in-house will probably interoperate better in the private cloud since the
third-party provider may not have a simple mechanism to integrate with these
custom applications. This is very common in the military space as in-house
applications are highly customized and often lack standards-based APIs. Some
cloud providers may not have this problem, but this depends entirely on their
network/application hosting software (OpenStack is one example discussed later
in this document). If the application is coded ``correctly'', APIs would be
exposed so that additional provider-hosted applications can integrate with the
in-house application. Too often, custom applications are written in a silo
where no such APIs are presented. \\

The table that follows compares access methods, reliability, and other
characteristics of the different cloud solutions.
% \begin{longtable}{p{2cm}p{3cm}p{3cm}p{3cm}p{3cm}}
\begin{longtable}{LLLLL}
  \toprule
  % top left square is blank
  &
  \textbf{Public Cloud}
  &
  \textbf{Private Cloud}
  &
  \textbf{Virtual Private Cloud}
  &
  \textbf{Inter-Cloud}
  \\ \midrule
  \textbf{Network Access}
  &
  Often times relies on Internet VPN, but could also use an Internet Exchange
  (IX) or private WAN
  &
  Corporate LAN or WAN, which is often private. Could be Internet-based if
  SD-WAN deployments (e.g. Viptela) are considered
  &
  Combination of corporate WAN for the private cloud components and whatever
  the public cloud access method is
  &
  Same as public cloud, except relies on the Internet as transport between
  clouds/cloud deployments
  \\ \midrule
  \textbf{Reliability and Accessibility}
  &
  Heavily dependent on highly-available and high-bandwidth links to the cloud
  provider
  &
  Often times high given the common usage of private WANs (backed by carrier SLAs)
  &
  Typically higher reliability to access the private WAN components, but
  depends entirely on the public cloud access method
  &
  Assuming applications are distributed, reliability can be quite high if at
  least one ``cloud'' is accessible (anycast)
  \\ \midrule
  \textbf{Fault Tolerance}
  &
  Typically high as the cloud provider is expected to have a highly redundant
  architecture based on cost
  &
  Often constrained by corporate CAPEX; tends to be a bit lower than a managed
  cloud service given the smaller DCs
  &
  Unlike public or private, the networking link between clouds is an important
  consideration for fault tolerance
  &
  Assuming applications are distributed, fault-tolerance can be quite high if
  at least one ``cloud'' is accessible (anycast)
  \\ \midrule
  \textbf{Performance}
  &
  Typically high as the cloud provider is expected to have a very dense
  compute/storage architecture
  &
  Often constrained by corporate CAPEX; tends to be a bit lower than a managed
  cloud service given the smaller DCs
  &
  Unlike public or private, the networking link between clouds is an important
  consideration, especially when applications are distributed across the two
  clouds
  &
  Unlike public or private, the networking link between clouds is an important
  consideration, especially when applications are distributed across the two
  clouds
  \\ \midrule
  \textbf{Scalability}
  &
  Appears to be ``infinite'' which allows the customer to provision new
  services quickly
  &
  High CAPEX and OPEX to expand it, which limits scale within a business
  &
  Scales well given public cloud resources
  &
  Highest; massively distributed architecture
  \\
  \bottomrule
  \caption{Cloud Design Comparison} \\
\end{longtable}
