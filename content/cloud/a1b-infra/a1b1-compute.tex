\subsection{Compute virtualization}
Conceptually, containers and virtual machines are similar in that they are a
way to virtualize services/machines on a single platform, effectively
achieving multi-tenancy. The subsections of this section will focus on their
differences and use cases, rather than discuss them at the top-level
section.
\\ \\
A brief discussion on two new design paradigms popular within any data center
is warranted. \textbf{Hyper-convergence and disaggregation} are polar opposites but are
both highly effective in solving specific business problems.
\\ \\
Hyper-convergence attempts to address issues with data center management and
resource consumption/provisioning. For example, the traditional DC
architecture will consist of four main components: network, storage, compute,
and services (firewalls, load balancers, etc.). These decoupled items could be
combined into a single and unified management infrastructure. The
virtualization and management layers are integrated into a single appliance,
and these appliances can be bolted together to scale-out linearly. Cisco
sometimes refers to this as the Lego block model. This reduces the capital
investments a business must make over time since the architecture need not
change as the business grows. Hyper-converged systems, by virtue of their
integrated management solution, simplify life cycle management of DC assets as
the ``single pane of glass'' concept can be used to manage all components.
Cisco's Hyperflex (also called Flexpod) is an example of a hyper-converged solution.
\\ \\
Disaggregation is the opposite of hyper-convergence in that rather than
combining functions (storage, network, and compute) into a single entity, it
breaks them apart even further. A network appliance, such as a router or
switch, can be decoupled from its network operating system (NOS). A white box
or brite box switch can be purchased at low cost with some other NOS
installed, such as Cumulus Linux. Cumulus generally does not sell hardware,
only a NOS, much like VMware. Server/computer disaggregation has been around
for decades since the introduction of the personal computer (PC) whereby the
common Microsoft Windows operating system was installed on machines from a
variety of manufacturers. Disaggregation in the network realm has been adopted
more slowly but has merit for the same reasons.
\subsubsection{Virtual Machines}
Virtual machine systems rely on a hypervisor, which is a software shim that
sits between the VMs themselves and the underlying hardware. The hardware
chipset would need to support this virtualization, which is a technique to
present hardware to VMs through the hypervisor. Each VM has its own OS which
is independent from the hypervisor. Hypervisors come in two flavors:

\begin{enumerate}
  \item \textbf{Type 1:} Runs on bare metal and is effectively an OS by
  itself. VMware ESXi and Linux Kernel-based  Virtual Machine (KVM) and are
  examples.
  \item \textbf{Type 2:} Requires an underlying OS and provides virtualization
  services on top through a hardware abstraction layer (HAL). VMware
  Workstation and VirtualBox are examples.
\end{enumerate}

VMs are considered quite heavyweight with respect to the overhead needed to
run them. This can reduce the efficiency of a hardware platform as the VM
count grows. It is especially inefficient when all of the VMs run the same OS
with very few differences other than configuration. A demonstration of
virtual machines is included in the NFVIS section of this document and is
focused on virtual network functions (VNF).

\subsubsection{Containers with Docker Demonstration}
Containers on a given machine all share the same OS, unlike with VMs. This
reduces the amount of overhead, such as idle memory taxes, storage space for
VM OS images, and the general maintenance associated with maintaining VMs.
Multi-tenancy is achieved by memory isolation, effectively segmenting the
different services deployed in different containers. There is still a thin
software shim between the underlying OS and the containers known as the
container manager, which enforces the multi-tenancy via memory isolation and
other techniques.
\\ \\
The main drawback of containers is that all containers must share the same OS.
For applications or services where such behavior is desired (for example, a
container per customer consuming a specific service), containers are a good
choice. As a general-purpose virtualization platform in environments where
requirements may change often (such as military networks), containers are a
poor choice.
\\ \\
Docker and Linux Containers (LXC) are popular examples of container engines.
The image that follow is from from www.docker.com that compares VMs to
containers at a high
level.

    \begin{minipage}[t]{\linewidth}
	  \centering
      \includegraphics[width=0.7\textwidth]{\imgpath docker-high-level.jpg}
      \captionof{figure}{Comparing Virtual Machines and Containers}
    \end{minipage}

This book does not detail the full Docker installation on CentOS because it is
already well-documented and not relevant to learning about containers. Once
Docker has been installed, run the following verification commands to ensure
it is functioning correctly. Any modern version of Docker is sufficient to
follow the example that will be discussed.

\begin{minted}{bash}
[centos@docker build]$ which docker && docker --version
/usr/bin/docker
Docker version 17.09.1-ce, build 19e2cf6
\end{minted}


Begin by running a new CentOS7 container. These images are stored on DockerHub
and are automatically downloaded when they are not locally present. For
example, this machine has not run any containers yet, and no images have been
explicitly downloaded. Thus, Docker is smart enough to pull the proper image
from DockerHub and spin up a new container. This only takes a few seconds on a
high-speed Internet connection. Once complete, Docker drops the user into a
new shell as the root user inside the container. The -i and -t options enable
an interactive TTY session, respectively, which is great for demonstrations.
Note that running Docker containers in the background is much more common as
there are typically many containers.

\begin{minted}{bash}
[centos@docker build]$ docker container run -it centos:7
Unable to find image 'centos:7' locally
7: Pulling from library/centos
469cfcc7a4b3: Pull complete 
Digest: sha256:989b936d56b1ace20ddf855a301741e52abca38286382cba7f44443210e96d16
Status: Downloaded newer image for centos:7

[root@088bbd2a7544 /]# 
\end{minted}

To verify that the correct container was downloaded, run the following
command. Then, exit from the container, as the only use for CentOS7 in our
example is to serve as a ``base'' image for the custom Ansible image to be
created.

\begin{minted}{bash}
[root@088bbd2a7544 /]# cat /etc/redhat-release 
CentOS Linux release 7.4.1708 (Core) 

[root@088bbd2a7544 /]# exit
\end{minted}

Exiting from the container effectively halts it, much like a process exiting
in Linux. Two interesting things have occurred. First, the image that was
downloaded is now stored locally in the image list. The image came from the
``centos'' repository with a tag of 7. Tags typically differentiate between
variants of a common image, such as version numbers or special features.
Second, the container list shows a CentOS7 container that recently exited.
Every container gets a random hexadecimal ID and random text names for
reference. The output can be very long, and so has been edited to fit the page
neatly.
 
\begin{minted}{bash}
[centos@docker build]$ docker image ls
REPOSITORY          TAG       IMAGE ID            CREATED             SIZE
centos              7         e934aafc2206        7 weeks ago         199MB

[centos@docker build]$ docker container ls -a
CONTAINER ID   IMAGE      COMMAND      CREATED         STATUS                 PORTS  NAMES
088bbd2a7544   centos:7   "/bin/bash"  1 minutes ago   Exited (0) 31 s ago    c      wise_banach
\end{minted}

To build a custom image, one creates a Dockerfile. It is a plain text file
that closely resembles a shell script and is designed to procedurally assemble
the required components of a container image for use later. The author already
created a Dockerfile using a CentOS7 image as a basic image and added some
additional features to it. Every step has been commented for clarity.

Dockerfiles are typically written to minimize the both number of ``layers'' and
amount of build time. Each instruction generally qualifies as a layer. The
more complex and less variable layers should be placed towards the top of the
Dockerfile, making them deeper layers. For example, installing key packages
and cloning the code necessary for the containers primary purpose occurs
early. Layers that are more likely to change, such as version-specific Ansible
environment setup parameters, can come later. This way, if the Ansible
environment changes and the image needs to be rebuilt, only the layers at or
after the point of modification must be rebuilt. The base CentOS7 image and
original yum package installations remain unchanged, substantially reducing
the image build time. Fewer RUN directives also results in fewer layers, which
explains the extensive use of \&\& and \ in the Dockerfile.

\begin{minted}{bash}
[centos@docker build]$ cat Dockerfile
# Start from CentOS 7 base image.
FROM centos:7

# Perform a number of shell commands to prepare the image:
#   * Update existing packages and install some new ones (alphabetical order)
#   * Clear the yum cache to reduce image size
#   * Minimally clone the specific branch to test
#   * Set up ansible environment
#   * Install PIP
#   * Install remaining ansible requirements through pip
RUN yum update -y && \
    yum install -y git \
                   tree \
                   which && \
    yum clean all && \
    \
    git clone \
        --branch command_authorization_failed_ios_regex \
        --depth 1 \
        --single-branch \
        --recursive \
        https://github.com/rcarrillocruz/ansible.git

# Setup the ansible environment and install dependencies via pip.
RUN /bin/bash -c "source /ansible/hacking/env-setup" && \
    echo "source /ansible/hacking/env-setup -q" >> /root/.bashrc && \
    \
    curl "https://bootstrap.pypa.io/get-pip.py" -o "get-pip.py" && \
    python get-pip.py && \
    rm -f get-pip.py && \
    \
    pip install -r /ansible/requirements.txt

# When starting a shell, start here to save a "cd" command.
# The ansible.cfg file, along with example inventories and playbooks,
# are located in this directory.
WORKDIR /ansible/examples

# Verify ansible on this image is functional for a "healthy" status.
# This only checks that the Ansible binary is in our PATH. A more interesting
# check could be running a simple Ansible playbook or "ansible -–version",
# but for this demo, the check is kept very basic.
HEALTHCHECK --interval=5m CMD which ansible || exit 1
\end{minted}

The Dockerfile is effectively a set of instructions used to build a custom
image. To build the image based on the Dockerfile, issue the command below.
The -t option specifies a tag, and in this case, ``cmd\_authz'' is used since
this particular Dockerfile is using a specific branch from a specific Ansible
developer's personal Github page. It would be unwise to call this simple
``ansible'' or ``ansible:latest'' due to the very specific nature of this
container and subsequent test. Because the user is in the same directory as
the Dockerfile, specify the ``.'' to choose the current directory. Each of the 5
steps in the Dockerfile (FROM, RUN, RUN, WORKDIR, and HEALTHCHECK) are logged
in the output below. The output looks almost identical to what one would see
through stdout.

\begin{minted}{bash}
[centos@docker build]$ docker image build -t ansible:cmd_authz .
Sending build context to Docker daemon  7.168kB
Step 1/5 : FROM centos:7
 ---> e934aafc2206
Step 2/5 : RUN yum update -y &&     yum install -y git                    tree                    which &&     yum clean all &&         git clone         --branch command_authorization_failed_ios_regex         --depth 1         --single-branch         --recursive         https://github.com/rcarrillocruz/ansible.git
 ---> Running in 84f969f5ee06
Loaded plugins: fastestmirror, ovl
Determining fastest mirrors
 * base: mirrors.lga7.us.voxel.net
 * extras: repo1.ash.innoscale.net
 * updates: repos-va.psychz.net
Resolving Dependencies
--> Running transaction check
---> Package acl.x86_64 0:2.2.51-12.el7 will be updated
[snip, many more packages]

Complete!
Loaded plugins: fastestmirror, ovl
Cleaning repos: base extras updates
Cleaning up everything
Cleaning up list of fastest mirrors

Cloning into 'ansible'...
 ---> b6b3ec4a0efb
Removing intermediate container 84f969f5ee06
Step 3/5 : RUN /bin/bash -c "source /ansible/hacking/env-setup" &&     echo "source /ansible/hacking/env-setup -q" >> /root/.bashrc &&         curl "https://bootstrap.pypa.io/get-pip.py" -o "get-pip.py" &&     python get-pip.py &&     rm -f get-pip.py &&         pip install -r /ansible/requirements.txt
 ---> Running in f8344dfe7384
[snip, progress messages]

Done!

  % Total    % Received % Xferd  Average Speed   Time    Time     Time  Current
                                 Dload  Upload   Total   Spent    Left  Speed
100 1603k  100 1603k    0     0  6836k      0 --:--:-- --:--:-- --:--:-- 6854k
Collecting pip
  Downloading https://files.pythonhosted.org/packages/0f/74/ecd13431bcc456ed390b44c8a6e917c1820365cbebcb6a8974d1cd045ab4/pip-10.0.1-py2.py3-none-any.whl (1.3MB)
Collecting setuptools
[snip, pip installations]
Successfully installed MarkupSafe-1.0 PyYAML-3.12 asn1crypto-0.24.0 bcrypt-3.1.4 cffi-1.11.5 cryptography-2.2.2 enum34-1.1.6 idna-2.6 ipaddress-1.0.22 jinja2-2.10 paramiko-2.4.1 pyasn1-0.4.3 pycparser-2.18 pynacl-1.2.1 six-1.11.0
 ---> 0cab65a61ee3
Removing intermediate container f8344dfe7384
Step 4/5 : WORKDIR /ansible/examples
 ---> 62ef1320c8da
Removing intermediate container f6b0e7ba51e1
Step 5/5 : HEALTHCHECK --interval=5m CMD which ansible || exit 1
 ---> Running in d17db16564d2
 ---> a8a6ac1b44e2
Removing intermediate container d17db16564d2
Successfully built a8a6ac1b44e2
Successfully tagged ansible:cmd_authz
\end{minted}

Once complete, there will be a new image in the image list. Note that there
are not any new containers, since this image has not been run yet. It is ready
to be instantiated as a container, or even pushed up to DockerHub for others
to use. Last, note that the container more than doubled in size. Because many
new packages were added for specific purposes, this makes the container less
portable. Smaller is always better, especially for generic images.

\begin{minted}{bash}
[centos@docker build]$ docker image ls
REPOSITORY   TAG           IMAGE ID            CREATED             SIZE
ansible      cmd_authz     a8a6ac1b44e2        2 minutes ago       524MB
centos       7             e934aafc2206        7 weeks ago         199MB
\end{minted}

For additional detail about this image, the following command returns
extensive data in JSON format. Docker uses a technique called layering whereby
each command in a Dockerfile is a layer, and making changes later in the
Dockerfile won't affect the lower layers. This is why the things least likely
to change should be placed towards the top, such as the base image, common
package installs, etc. This reduces image building time when Dockerfiles are
changed.

\begin{minted}{bash}
[centos@docker build]$ docker image inspect a8a6ac1b44e2 | head -5
[
    {
        "Id": "sha256:a8a6ac1b44e28f654572bfc57761aabb5a92019c[snip]",
        "RepoTags": [
            "ansible:cmd_authz"
[snip]
\end{minted}

To run a container, use the same command shown earlier to start the CentOS7
container. Specify the image name and in less than second, the new container
is 100\% operational. Ansible should be installed on this container as part of
the image creation process, so be sure to test this. Running the ``setup''
module on the control machine (the container itself) should yield several
lines of JSON output about the device itself. Note that, towards the bottom of
this output dump, ansible is aware that it is inside a Docker container.

\begin{minted}{bash}
[centos@docker build]$ docker container run -it ansible:cmd_authz
[root@04eb3ee71a52 examples]# which ansible && ansible -m setup localhost 
/ansible/bin/ansible
localhost | SUCCESS => {
    "ansible_facts": {
        [snip, lots of information]
        "ansible_virtualization_type": "docker", 
        "gather_subset": [
            "all"
        ], 
        "module_setup": true
    }, 
    "changed": false
}
\end{minted}

Next, create the playbook used to test the specific issue. The full playbook
is shown below. For those not familiar with Ansible at all, please see the
Ansible demonstration in this book, or go to the author's Github page for many
production-quality examples. This 3 step playbook is simple:

\begin{enumerate}
  \item Define the login credentials so Ansible can log into the router.
  \item Log into the router, enter configuration mode, and run ``do show
  clock''. Store the output.
  \item	Print out the value of the output variable and look for the date/time
  in the JSON structure.
\end{enumerate}

\begin{minted}{yaml}
[root@04eb3ee71a52 examples]# cat issue31575.yml
---
- hosts: csr1.njrusmc.net
  gather_facts: false
  connection: network_cli
  tasks:
    - name: "SYS >> Define router credentials"
      set_fact:
        provider:
          host: "{{ inventory_hostname }}" 
          username: "ansible"
          password: "ansible"

    - name: "IOS >> Run show command from config mode" 
      ios_config:
        provider: "{{ provider }}"
        commands: "do show clock"
        match: none
      register: output

    - name: "DEBUG >> Print output"
      debug:
        var: output
...
\end{minted}

Before running this playbook, a few Ansible adjustments are needed. First,
adjust the ansible.cfg file to use the hosts.yml inventory file and disable
host key checking. Ansible needs to know which network devices are in its
inventory and how to handle unknown SSH keys.

\begin{minted}{bash}
[root@04eb3ee71a52 examples]# head -20 ansible.cfg 
[snip, comments]
[defaults]

# some basic default values...

inventory         = hosts.yml
host_key_checking = False
\end{minted}

Next, ensure the inventory contains the specific router in question. In this
case, it is a Cisco CSR1000v running in AWS. Note that we would have used
``echo'' commands in our Dockerfile to address these issues in advance, but this
specific information makes the docker image less useful and less portable.

\begin{minted}{yaml}
[root@04eb3ee71a52 examples]# head -20 hosts.yml
---
# This is the default ansible 'hosts' file.
#
# It should live in /etc/ansible/hosts
all:
  hosts:
    csr1.njrusmc.net
\end{minted}

Before connecting, ensure your container can use DNS to resolve the IP address
for the router's hostname (assuming you are using DNS), and ensure the
container can ping the router. This rules out any networking problems. The
author does not show the initial setup of the CSR1000v, which includes adding
a username/password of ansible/ansible, and nothing else.

\begin{minted}{bash}
[root@04eb3ee71a52 examples]# ping –c 3 csr1.njrusmc.net
PING csr1.njrusmc.net (18.x.x.x) 56(84) bytes of data.
64 bytes from ec2-18-x-x-x.x.com (18.x.x.x): icmp_seq=1 ttl=253 time=0.884 ms
64 bytes from ec2-18-x-x-x.x.com (18.x.x.x): icmp_seq=2 ttl=253 time=1.03 ms
64 bytes from ec2-18-x-x-x.x.com (18.x.x.x): icmp_seq=3 ttl=253 time=0.971 ms

--- csr1.njrusmc.net ping statistics ---
3 packets transmitted, 3 received, 0% packet loss, time 2002ms
\end{minted}

The last step executes the playbook from inside the container. This
illustrates the original issue that the ios\_config module, at the time of this
writing, does not return device output. The author's personal preference is to
always print the Ansible version number before running playbooks designed to
test issues. This reduces the likelihood of invalid test results due to
version confusion. In the DEBUG step below, there is no date/time output,
which helps illustrate the Ansible issue that is being investigated.

\begin{minted}{bash}
[root@9bc07956b416 examples]# ansible --version | head -1
ansible 2.6.0dev0 (command_authorization_failed_ios_regex 5a1568c753) last updated 2018/05/28 17:19:42 (GMT +000)

[root@04eb3ee71a52 examples]# ansible-playbook issue31575.yml 

PLAY [csr1.njrusmc.net] **************************************

TASK [SYS >> Define router credentials] **********************
ok: [csr1.njrusmc.net]

TASK [IOS >> Run show command from config mode] **************
changed: [csr1.njrusmc.net]

TASK [DEBUG >> Print output] *********************************
ok: [csr1.njrusmc.net] => {
    "output": {
        "banners": {}, 
        "changed": true, 
        "commands": [
            "do show clock"
        ], 
        "failed": false, 
        "updates": [
            "do show clock"
        ]
    }
}

PLAY RECAP ****************************************************
csr1.njrusmc.net           : ok=3    changed=1    unreachable=0    failed=0   
\end{minted}

After exiting this container, check the list of containers again. Now, there
were 2 containers in the past, the newest one at the top. This was the Ansible
container we just exited after completing our test. Again, some output has
been truncated to make the table fit neatly.

\begin{minted}{bash}
[centos@docker build]$ docker container ls -a
CONTAINER ID    IMAGE           COMMAND      CREATED    STATUS                 PORTS           NAMES
04eb3ee71a52    ans:cmd_authz   "/bin/bash"  33 m ago   Exited (127) 7 s ago   adoring_mestorf
088bbd2a7544    centos:7        "/bin/bash"  43 m ago   Exited (0) 42 m ago    wise_banach
\end{minted}

This manual ``start and stop'' approach to containerization has several
drawbacks. Two are listed below:
\begin{enumerate}
  \item	To retest this solution, the playbook would have to be created again,
  and the Ansible environment files (ansible.cfg, hosts.yml) would need to be
  updated again. Because containers are ephemeral, this information is not
  stored automatically.
  \item	The commands are difficult to remember and it can be a lot to type,
  especially when starting many containers. Since containers were designed for
  microservices and expected to be deployed in dependent groups, this
  management strategy scales poorly.
\end{enumerate}
  
Docker includes a feature called docker-compose. Using YAML syntax, developers
can specify all the containers they want to start, along with any minor
options for those containers, then execute the compose file like a script. It
is better than a shell script since it is more portable and easier to read. It
is also an easy way to add volumes to Docker. There are different kinds of
volumes, but in short, volumes allow persistent data to be passed into and
retrieved from containers. In this example, a simple directory mapping (known
as a ``bind mount'' in Docker) is built from the local mnt\_files/ folder to the
container's file system. In this folder, one can copy the Ansible files
(issue31575.yml, ansible.cfg, and hosts.yml) so the container has immediate
access. While it is possible to handle volume mounting from the commands
viewed previously, it is tedious and complex.

\begin{minted}{yaml}
[centos@docker compose]$ cat docker-compose.yml 
version: '3.2'
services:
  ansible:
    image: ansible:cmd_authz
    hostname: cmd_authz
    # Next two lines are equivalent of -i and -t, respectively
    stdin_open: true
    tty: true
    volumes:
      - type: bind
        source: ./mnt_files
        target: /ansible/examples/mnt_files
\end{minted}

The contents of these files was shown earlier, but ensure they are all placed
in the mnt\_files/ directory with relation to where the docker-compose.yml file
is located.

\begin{minted}{bash}
[centos@docker compose]$ tree
.
├── docker-compose.yml
└── mnt_files
    ├── ansible.cfg
    ├── hosts.yml
    └── issue31575.yml
\end{minted}

To run the docker-compose file, use the command below. It will build
containers for all keys specified under the ``services'' dictionary. In this
case, there is only one container called ``ansible'' which is based on the
``ansible:cmd\_authz'' image created earlier from the custom Dockerfile. The -i
and -t options are enabled to allow for interactive shell access. The -d
option with the docker-compose command specifies the ``detach'' operation, which
runs the containers in the background. View the list of containers to see the
new Ansible container running successfully.

\begin{minted}{bash}
[centos@docker compose]$ docker-compose up -d
Starting compose_ansible_1 ... done

[centos@docker compose]$ docker container ls
CONTAINER ID   IMAGE           COMMAND       CREATED    STATUS            PORTS NAMES
d3f1365f3145   ans:cmd_authz   "/bin/bash"    1 m ago   Up 32 s (health: ...)   compose_ansible_1
\end{minted}

The command below says ``execute, on the ansible container, the bash command''
which grants shell access. Ensure that the mnt\_files/ directory exists and
contains all the necessary files. Copy the contents to the current directly,
which will overwrite the basic ansible.cfg and hosts.yml files provided by
Ansible.

\begin{minted}{bash}
[centos@docker compose]$ docker-compose exec ansible bash
[root@cmd_authz examples]# tree mnt_files/
mnt_files/
|-- ansible.cfg
|-- hosts.yml
`-- issue31575.yml

[root@cmd_authz examples]# cp mnt_files/* .
cp: overwrite './ansible.cfg'? y
cp: overwrite './hosts.yml'? y
\end{minted}

Run the playbook again, and observe the same results as before. Now, assuming
that this issue remains open for a long period of time, docker-compose helps
reduce the test setup time.

\begin{minted}{bash}
[root@cmd_authz examples]# ansible-playbook issue31575.yml 

PLAY [csr1.njrusmc.net] ****************************************

TASK [SYS >> Define router credentials] ************************
[snip]
\end{minted}

Exit from the container and check the container list again. Notice that,
despite exiting, the container continues to run. This is because
docker-compose created the container in a detached state, meaning the absence
of the shell does not cause the container to stop. Manually stop the container
using the commands below. Note that only the first few characters of the
container ID can be used for these operations.

\begin{minted}{bash}
[centos@docker compose]$ docker container ls -a
CONTAINER ID  IMAGE              COMMAND     CREATED    STATUS          PORTS NAMES
c16452e2a6b4  ansible:cmd_authz  "/bin/bash" 12 m ago  Up 10 m (health: ...)  compose_ansible_1
04eb3ee71a52  ansible:cmd_authz  "/bin/bash" 2 h ago   Exited (127) 2 h ago   adoring_mestorf
088bbd2a7544  centos:7           "/bin/bash" 2 h ago   Exited (0) 2 h ago     wise_banach

[centos@docker compose]$ docker container stop c16
c16

[centos@docker compose]$ docker container ls -a
CONTAINER ID  IMAGE              COMMAND     CREATED   STATUS          PORTS NAMES
c16452e2a6b4  ansible:cmd_authz  "/bin/bash" 12 m ago  Exited (137) 1 m ago  compose_ansible_1
04eb3ee71a52  ansible:cmd_authz  "/bin/bash" 2 h ago   Exited (127) 2 h ago  adoring_mestorf
088bbd2a7544  centos:7           "/bin/bash" 2 h ago   Exited (0) 2 h ago    wise_banach
\end{minted}

For total cleanup, delete these stale containers from the demonstration so
that they are not accidentally used for future use. Remember, containers are
ephemeral, and should be built and discarded regularly.


\begin{minted}{bash}
[centos@docker compose]$ docker container rm c16 04e 088
c16
04e
088
[centos@docker compose]$ docker container ls -a
CONTAINER ID  IMAGE   COMMAND   CREATED   STATUS     PORTS     NAMES
[no further output]
\end{minted}

\subsubsection{Python Virtual Environments (venv) for Refactoring}
