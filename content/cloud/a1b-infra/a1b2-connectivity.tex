\subsection{Connectivity}
Network virtualization is often misunderstood as being something as simple as
``virtualize this device using a hypervisor and extend some VLANs to the host''.
Network virtualization is really referring to the creation of virtual
topologies using a variety of technologies to achieve a given business goal.
Sometimes these virtual topologies are overlays, sometimes they are forms of
multiplexing, and sometimes they are a combination of the two. Here are some
common examples (not a complete list) of network virtualization using
well-known technologies. Before discussing specific technical topics like
virtual switches and SDN, it is worth discussing basic virtualization
techniques upon which all of these solutions rely.

\begin{enumerate}
  \item \textbf{Ethernet VLANs using 802.1q encapsulation.} Often used to create
  virtual networks at layer 2 for security segmentation, traffic hair pinning
  through a service chain, etc. This is a form of data multiplexing over
  Ethernet links. It isn’t a tunnel/overlay since the layer 2 reachability
  information (MAC address) remains exposed and used for forwarding decisions.
  \item \textbf{VPN Routing and Forwarding (VRF) tables or other layer-3
  virtualization techniques.} Similar uses as VLANs except virtualizes an
  entire routing instance, and is often used to solve a similar set of
  problems. Can be combined with VLANs to provide a complete virtual network
  between layers 2 and 3. Can be coupled with GRE for longer-range
  virtualization solutions over a core network that may or may not have any
  kind of virtualization. This is a multiplexing technique as well but is
  control-plane only since there is no change to the packets on the wire, nor
  is there any inherent encapsulation (not an overlay).
  \item \textbf{Frame Relay DLCI encapsulation.} Like a VLAN, creates segmentation
  at layer 2 which might be useful for last-mile access circuits between PE and
  CE for service multiplexing. The same is true for Ethernet VLANs when using
  EV services such as EV-LINE, EV-LAN, and EV-TREE\@. This is a data-plane
  multiplexing technique specific to Frame Relay.
  \item \textbf{MPLS VPNs.} Different VPN customers, whether at layer 2 or layer 3,
  are kept completely isolated by being placed in a separate virtual overlay
  across a common core that has no/little native virtualization. This is an
  example of an overlay type of virtual network.
  \item \textbf{Virtual eXtensible Area Network (VXLAN).} Just like MPLS VPNs;
  creates virtual overlays atop a potentially non-virtualized core. VXLAN is a
  MAC-in-IP/UDP tunneling encapsulation designed to provide layer-2 mobility
  across a data center fabric with an IP-based underlay network. The advantage
  is that the large layer-2 domain, while it still exists, is limited to the
  edges of the network, not the core. VXLAN by itself uses a ``flood and learn''
  strategy so that the layer-2 edge devices can learn the MAC addresses from
  remote edge devices, much like classic Ethernet switching. This is not a
  good solution for large fabrics where layer-2 mobility is required, so VXLAN
  can be paired with BGP’s Ethernet VPN (EVPN) address family to provide MAC
  routing between endpoints. Being UDP-based, the VXLAN source ports can be
  varied per flow to provide better underlay (core IP transport) load
  sharing/multipath routing, if required.
  \item \textbf{Network Virtualization using Generic Routing Encapsulation
  (NVGRE).} This technology extends classic GRE tunneling to include a subnet
  identifier within the GRE header, allowing GRE to tunnel layer-2 Ethernet
  frames over IP/GRE\@. The use cases for NVGRE are also identical to VXLAN
  except that, being a GRE packet, layer-4 port-based load sharing is not
  supported. Some devices can support GRE key-based hashing, but this does not
  have flow-level visibility.
  \item \textbf{OTV.} Just like MPLS VPNs; creates virtual overlays atop a
  potentially non-virtualized core, except provides a control-plane for MAC
  routing. IP multicast traffic is also routed intelligently using GRE
  encapsulation with multicast destination addresses. This is another example
  of an overlay type of virtual network.
\end{enumerate}

\subsubsection{Virtual Switches}
The term ``virtual switch'' has multiple meanings. As discussed in the previous
section, the most generic interpretation of the term would be ``VLAN''. A VLAN
is, quite literally, a virtual switch, which shares the same hardware as all
the other VLANs next to it, but remains logically isolated. Along these lines,
a VRF is a virtual router and a Cisco ASA context is a virtual firewall.

However, it is likely that this section of the Evolving Technologies blueprint
is more interested in discussing virtual switches in the context of
hypervisors. Simply put, a virtual switch serves as a bridge between the
applications and the physical network. Virtual machines map their virtual NICs
to the virtual switch ports, much like a physical server connects into a data
center access switch. The virtual switches are also connected to the physical
server NICs, often times with 802.1q VLAN trunking enabled, just like a real
switch. Each port (or group of ports) can map to a single VLAN, providing
VLAN-tagged transport to the physical network and untagged transport to the
applications, as expected. Some engineers prefer to think about virtual
switches as the true access switch in the network, with the top of rack (TOR)
switch being an aggregation device of sorts.

There are many types of virtual switches:
\begin{enumerate}
  \item \textbf{Standalone/basic:} As described above, these switches support
  basic features such as access ports, trunk ports, and some basic security
  settings such as policing, and MAC spoof protection. They are independently
  managed on each server, and while simple to build, they become difficult to
  maintain as the data center computing environment scales.
  \item \textbf{Distributed:} A distributed virtual switch is managed as a
  single entity despite being spread across many servers. Loosely analogous to
  Cisco StackWise or Virtual Switching System (VSS) technologies, this reduces
  the management burden. The individual servers still have local switches that
  can tolerate a management outage, but are centrally managed. Distributed
  virtual switches tend to have more features than standalone ones, such as
  LACP, QoS, private VLANs, Netflow, and more. VMware's distribution virtual
  switch (DVS) is available in vCenter-enabled deployments and is one such example.
  \item \textbf{ Vendor-specific software:} Several vendors offer
  software-based virtual switches with comprehensive feature sets. Cisco's
  Nexus 1000v, for example, is one such product. These solutions typically
  offer strong CLI/API support for better integration into a uniform
  management strategy. Other solutions may even be integrated with the
  hypervisor's management system despite being add-on products. Many modern
  virtual switches can, for example, terminate VXLAN tunnels. This brings
  multi-tenancy all the way to the server without introducing the complexity
  into the data center physical switches.
\end{enumerate}

\subsubsection{Software-Defined Wide Area Network (SD-WAN Viptela Demonstration)}
The Viptela SD-WAN solution provides a highly capable and adaptive WAN
solution to help customers reduce WAN costs (OPEX and CAPEX), gain additional
performance/monitoring insight, and optimize performance. It has four main
components:

\begin{enumerate}
  \item \textbf{vSmart Controller:} The centralized control-plane and policy
  injection service for the network.
  \item \textbf{vEdge:} The branch device that registers to the vSmart
  controllers to receive policy updates. Each vEdge router requires about 100
  kbpsof bandwidth back to the vSmart controller.
  \item \textbf{vManage:} The single-pane-of-glass management front-end that
  provides visibility, analytics, and easy policy adjustment.
  \item \textbf{vBond:} Technology used for Zero Touch Provisioning (ZTP),
  enabling the vEdge devices to discover available vSmart controllers. This
  component is effectively a communications broker between SD-WAN endpoints
  (vEdge) and their controllers (vSmart).
\end{enumerate}

The control-plane is TLS-based and is formed between vEdge devices and vSmart
controllers. The digital certificates for Viptela’s PKI solution are internal
and easily managed within vManage; a complex, preexisting PKI is not
necessary. The routing design is similar in logic to BGP route-reflectors
whereby individual vEdge devices can send traffic directly between one another
without directly exchanging any reachability/policy information. To provide
high-scale network services, the Overlay Management Protocol (OMP) is a
BGP-like protocol that carries a variety of attributes. These attributes
include application/QoS specific routing policy, multicast routing
information, IPsec keys, and more.

The solution supports both IPsec ESP and GRE data-plane encapsulations for its
overlay networks. Because OMP carries IPsec keys within the system’s
control-plane, Internet Key Exchange (IKE) between vEdge endpoints is
unnecessary. This optimization obviates the need for IKE, reducing both vEdge
device state and spoke-to-spoke tunnel setup time.

Like many SD-WAN solutions, Viptela can classify traffic based on traditional
mechanisms such as ports, protocols, IP addresses, and DSCP values. It can
also perform application-specific classification with policies tuned for each
specific application. All policies are configured through the vManage
interface which are then communicated to the controller. The controller then
communicates this to the vEdge devices.

Although the definitions are imperfect, it is mostly correct to say that the
vManage-to-vSmart controller interface is a northbound interface (except that
vManage is a management console, not a business application). Likewise, the
vSmart-to-vEdge interface is like a southbound interface. Also note that,
unlike truly centralized control planes, the failure of a vSmart controller or
the path by which a vEdge uses to reach a vSmart controller results in the
vEdge reverting back to the last applied policy. This means that the WAN can
function like a distributed control-plane provided changes are not needed. As
such, the Viptela solution can be generally classified as a hybrid SDN solution.

ZTP relies on vBond, which is an orchestration process that allows vEdge
devices to join the SD-WAN instance without any pre-configuration on the
remote devices. Each device comes with an embedded SSL certificate stored
within a Trusted Platform Module (TPM). Via vManage, the network administrator
can trust or not trust this particular client device. Revoking trust for a
device is useful for cases where the vEdge is lost or stolen, much like
issuing a Certificate Revocation List (CRL). Once the trust settings are
updated, vManage notifies the vSmart controllers so they can accept or reject
the SSL sessions from vEdge devices.

The Viptela SD-WAN solution also supports network-based multi-tenancy. A
4-byte shim header called a label (not to be confused with MPLS labels) is
added to each packet within a specific tenant’s overlay as a membership
identifier. As such, Viptela can tie into existing networks using technologies
like VRF + VLAN in a back-to-back fashion, much like Inter-AS MPLS Option A
(RFC 4364 Section 10a). The diagram that follows summarizes Viptela at a high level.

\addimg{viptela-hl.png}{0.7}{Viptela SD-WAN High Level}

The remainder of this section walks through a high-level demonstration of the
Viptela SD-WAN solution's various interfaces. Upon login to vManage, the
centralized and multi-tenant management system, the user is presented with a
comprehensive dashboard. At its most basic, the dashboard alerts the
administrator to any obvious issues, such as sites being down or other errors
needing repair.

\addimg{viptela-home.png}{0.7}{Viptela Home Dashboard}

Clicking on the vEdge number ``4'', one can explore the status of the four
remote sites. While not particularly interesting in a network where everything
is working, this provides additional details about the sites in the network,
and is a good place to start troubleshooting when issues arise.

\addimg{viptela-nodes.png}{0.7}{Viptela Node Summary}

Next, the administrator can investigate a specific node in greater detail to
identify any faults recorded in the event log. The screenshot on the following
page is from SDWAN4, which provides a visual representation of the current
events and the text details in one screen.

\addimg{viptela-events.png}{0.7}{Viptela Event Logging}

The screenshot below depicts the bandwidth consumed between different hosts on
the network. More granular details such as ports, protocols, and IP addresses
are available between the different monitoring options from the left-hand
pane. This screenshot provides output from the ``Flows'' option on the SDWAN4
node, which is a physical vEdge-100m appliance.

\addimg{viptela-flows.png}{0.7}{Viptela Flow Exploration}

Last, the solution allows for granular flow-based policy control, similar to
traditional policy-based routing, except centrally controlled and fully
dynamic. The screenshot below shows a policy to match DSCP 46, typically used
for expedited forwarding of inelastic, interactive VOIP traffic. The preferred
color (preferred link in this case) is the direct Ethernet-based Internet
connection this particular node has. Not shown is the backup 4G LTE link this
vEdge-100m node also has. This link is slower, higher latency, and less
preferable for voice transport, so we administratively prefer the wireline
Internet link. Not shown is the SLA configuration and other policy parameters
to specify the voice performance characteristics that must be met. For
example: 150 ms one way latency, less than 0.1\% packet loss, and less than 30
ms jitter. If the wireline Internet exceeds any of these thresholds, the
vSmart controllers with automatically start using the 4G LTE link, assuming
that its performance is within the SLA's specification.

\addimg{viptela-voice-policy.png}{0.7}{Viptela VoIP QoS Policy}

For those interested in replicating this demonstration, please visit
\href{https://dcloud.cisco.com/}{Cisco dCloud}.
Note that the compute/storage requirements for these Cisco SD-WAN components
is very low, making it easy to run almost anywhere. The only exception is the
vManage component and its VM requirements can be found
\href{https://sdwan-docs.cisco.com/Product_Documentation/Getting_Started/
Hardware_and_Software_Installation/Server_Hardware_Recommendations}{here}.
The VMs can be run either on VMware ESXi or Linux KVM-based hypervisors (which
includes Cisco NFVIS discussed later in this book).

\subsubsection{Software-Defined Access (SDA)}
Cisco's SDA architecture is a holistic, intent-based networking solution
designed for enterprises to operate, maintain, and secure their access layer
networks. Campus Fabric is one of the core components of this design, and is
of particular interest to network engineers.

Cisco’s Campus Fabric is a main component of the Digital Network Architecture
(DNA), a major Cisco networking initiative. Campus Fabric relies on a
VXLAN-based data plane, encapsulating traffic at the edges of the fabric
inside IP packets to provide L2VPN and L3VPN service. Security Group Tags
(SGT), Quality of Service (QoS) markings, and the VXLAN Virtual Network ID
(VNI) are all carried in the VXLAN header, giving the underlay network some
ability to apply policy to transit traffic. Campus Fabric was designed with
mobility, scale, and performance in mind.

The solution uses Location/Identification Separation Protocol (LISP) as its
control-plane. LISP is like a combination of DNS and NHRP as a mapping server
binds endpoint IDs (EIDs) to routing locations (RLOCs) in a centralized
manner. Like NHRP, LISP is a reactive control plane whereby EIDs are exchanged
between endpoints via ``conversational learning''. That is to say, edge nodes
don’t retain all state at all times, but rather only when it is needed. The
initial setup of communications between two nodes when the state is absent can
take some time as LISP converges. Unlike DNS, the LISP mapping server does not
reply directly to LISP edge nodes as such a reply is not a guarantee that two
edge nodes can actually communicate. The LISP mapping server forwards the
request to the remote edge node authoritative for a given EID, which generates
the response. This behavior is similar to how NHRP works in DMVPN phases 2 and
3 when spoke-to-spoke tunnels are dynamically built.

Campus Fabric offers separation using both policy-based segmentation via
Security Group Tags (SGT) and network-based segmentation via VXLAN/LISP\@. These
options are not mutually exclusive and can be deployed together for even
better separation between virtual networks. Extending virtual networks outside
of the fabric is done using VRF-Lite in an MPLS Inter-AS Option A fashion,
effectively extending the virtual networks without merging the control-planes.
This architecture can be thought of like an SD-LAN although Cisco (and the
industry in general) do not use this term. The IP routed underlay is kept
simple with complex, tenant-specific overlays added on top according to the
business needs.

Note that Campus Fabric is the LAN networking component of the overall SDA
architecture. Fabric border nodes are similar to fabric edge nodes (access
switches) except that they connect the fabric to upstream resources, such as
the core network. This is how users access other places in the network, such
as WAN, data center, Internet edge, etc. DNA-C, ISE, Network Data Platform
(NDP) analytics, and wireless LAN controllers (WLC) are typically locally in
the data center and control the entire SDA architecture. Being able to express
intent in human language through DNA-C, then have the system map this intent
to device configuration automatically, is a major advantage SDA deployment.

At the time of this writing, the SDA solution is supported on most modern
Cisco switch product lines. Some of the common ones include the Catalyst 9000,
Catalyst 3650/3850, and Nexus 7000 lines. DNA-C within the SDA architecture is
analogous to an SDN controller as it asserts the desired configuration/state
onto the managed devices within the SDA architecture. DNA-C is programmable
through a northbound REST API to allow business applications to communicate
their intent to DNA-C, which uses its southbound interfaces (SSH, SNMP, and/or
HTTPS) to program network devices.

\subsubsection{Software-Defined Data Center (SD-DC)}
SD-DC as a generic term describes a data center design model whereby all DC
resources are virtualized and on-demand. That is to say, SD-DC brings
cloud-like provisioning of all DC resources (compute, network, and storage) to
support specific applications in automated fashion. Security within
application components (e.g.\ front-end, application, and database containers)
and between different applications (e.g. APIs) is inherent with any SD-DC
solution. Resources are pooled and shared between resources for maximum cost
effectiveness, flexibility, and ability to respond to changing market demands.

One example of an SD-DC solution is Cisco's Application Centric Infrastructure
(ACI). As discussed earlier, ACI separates policy from reachability and could
be considered a hybrid SDN solution, much like Cisco's original SD-WAN
solution, Intelligent WAN (IWAN). ACI is more revolutionary than IWAN as it
reuses less technology and relies on custom hardware and software.
Specifically, ACI is supported on the Cisco Nexus 9000 product line using a
version of software specific to ACI\@. This differs from the original NX-OS
which is considered a ``standalone'' or ``non-ACI'' deployment of the Nexus 9000.
Unlike IWAN, ACI is positioned within the data center as a software defined
data center (SDDC) solution.

The creation of ACI, to include its complement of customized hardware, was
driven by a number of factors (not a comprehensive list):

\begin{enumerate}
  \item Software alone cannot solve the migration of 1Gbps to 10Gbps in the
  server access layer or 10Gbps to 40Gbps/100Gbps in the DC aggregation and
  core layers.
  \item The overall design of the DC has to change to better support east/west
  (lateral flows within the DC) traffic flows being generated by distributed,
  multi-tiered applications. Traditional DC designs focused on north/south
  (into and out of the DC) traffic for user application access.
  \item There is a need for rapid service deployment for internal IT consumers
  in a secure and scalable way. This prevents individuals from going
  ``elsewhere'' (unauthorized third-parties, for example) when enterprise IT
  providers cannot meet their needs.
  \item Central management isn’t a new thing, but has traditionally failed as
  network devices did not have machine-friendly interfaces since they were
  often configured directly by humans (SNMP is an exception). Such interfaces
  are called Application Programmability Interfaces (API) which are discussed
  later in this document.
\end{enumerate}

The controller used for ACI is known as the Application Policy Infrastructure
Controller (APIC). Cisco’s approach in developing this controller was
different from the classic thought process of ``the controller needs to get a
packet from point A to point B''. Networks have traditionally done this job
well. Instead, APIC focuses on \textit{when the packets can move and what
happens when they do}. That is to say, under what policy conditions a
packet should be forward, dropped, rerouted over an alternative link, etc.
Packet forwarding continues in the distributed control-plane model as
discussed before, but the APIC is able to configure any node in the network
with specific policy, to include security policy, or to enhance/modify any
given flow in the data center. Policy is retained in the nodes even in the
event of a controller failure, but policy can only be modified by the APIC
control point.

The ACI infrastructure is built on a leaf/spine network fabric which has a
number of interesting characteristics:

\begin{enumerate}
  \item Adding bandwidth is achieved by adding spines and linking the leaves to it.
  \item Adding access density is achieved by adding leaves and linking them to
  the spines.
  \item ``Border'' leaves are identified as the egress point from the fabric,
  not the spines.
  \item Nothing connects to the spines other than leaves (no services).
  \item Spines are never connected laterally.
\end{enumerate}

This architecture is not new (in general), but is becoming popular in DC
fabrics for its superior distribution of bandwidth for north/south and
east/west DC flows. The significant advantage of this design for any SDN DC
solution is that it is universally useful; other SDN vendors in the DC space
typically prefer that the underlying architecture look this way. This topology
need not change even as the APIC policies change significantly since it is
designed only for high-speed transport. In an ACI network, the network makes
no attempt to automatically classify, treat, and prioritize specific
applications absent input from the user (via APIC). That is to say, it is both
cost prohibitive and error prone for the network to make such a classification
when the business drivers (i.e.\ human input) are what drives the
prioritization, security policy, and other treatment characteristics of a
given application's traffic.

Policy applied to the APIC is applied to the network using several constructs:

\begin{enumerate}
  \item \textbf{Application Network Profile:} Logical template for how the
  application connects and works. All tiers of a given application are
  encompassed by this profile. The profile can contain multiple policies which
  are applied between the components of an application. These policies can
  define things like QoS, availability, and security requirements. This
  ``declarative'' model is intuitive and is application-focused. The policy
  also follows applications across the DC as they migrate for mobility purposes.
  \item \textbf{Endpoint Groups (EPG):} EPGs are designed to group elements
  that share a common policy together. Consider a classic three-tier
  application. All web servers would be in one EPG, while the application
  servers would be in a second. The database servers would be in a third. The
  policy application would occur between these endpoint groups. Components are
  placed into groups based on any number of fields, such as VLAN, IP address,
  port (physical or layer-4), and other fields.
  \item \textbf{Contracts:} Contracts determine the types of traffic (and
  their treatment) between EPGs. The contract is the application of policy
  between EPGs which effectively represents an agreement between two entities
  to exchange information. Contracts are aptly named as it is not possible to
  violate a contract; this is enforced by the APIC policies, which are driven
  by business requirements.
\end{enumerate}

The diagram below depicts a high level image of the ACI infrastructure provided by Cisco.

\addimg{aci-overview.png}{0.7}{Cisco ACI SD-DC High Level}
