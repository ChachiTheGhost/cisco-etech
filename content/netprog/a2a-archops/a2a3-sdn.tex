\subsection{Controller based network design}
Software-Defined Networking (SDN) is a concept that networks can be both
programmable and disaggregated concurrently, ultimately providing additional
flexibility, intelligence, and customization for the network administrators.
Because the definition of SDN varies so widely within the network community,
it should be thought of as a continuum of different models rather than a
single, prescriptive solution. \\

There are four main SDN models as defined in
\href{http://www.ciscopress.com/store/art-of-network-architecture-business-driven-design-9780133259230}{The Art of Network Architecture: Business-Driven Design} by
\href{https://twitter.com/rtggeek}{Russ White} and
\href{ihttps://twitter.com/LadyNetwkr}{Denise Donohue} (Cisco Press 2014).
The models are discussed briefly below.

\begin{enumerate}
  \item \textbf{Distributed:} Although not really an ``SDN'' model at all, it
  is important to understand the status quo. Network devices today each have
  their own control-plane components which rely on distributed routing
  protocols (such as OSPF, BGP, etc). These protocols form paths in the
  network between all relevant endpoints (IP prefixes, etc). Devices typically
  do not influence one another’s routing decisions individually as traffic is
  routed hop-by-hop through the network without centralized oversight. This
  model totally distributes the control-plane across all devices. Such
  control-planes are also autonomous; with minimal administrative effort, they
  often form neighborships and advertise topology and/or reachability
  information. Some of the drawbacks include potential routing loops
  (typically transient ones during periods of convergence) and complex routing
  schemes in poorly designed/implemented networks. The diagram that follows depicts
  several routers each with their own control-plane and no centralization.

  \addimg{sdn-distributed.jpg}{0.7}{SDN Model --- Distributed}

  \item \textbf{Augmented:} This model relies on a fully distributed
  control-plane by adding a centralized controller that can apply policy to
  parts of the network at will. Such a controller could inject shorter-match
  IP prefixes, policy-based routing (PBR), security features (ACL), or other
  policy objects. This model is a good compromise between distributing
  intelligence between nodes to prevent singles points of failure (which a
  controller introduces) by using a known-good distributed control-plane
  underneath. The policy injection only happens when it ``needs to'', such as
  offloading traffic from an overloaded link in a DC fabric or traffic from a
  long-haul fiber link between two points of presence (POPs) in an SP core.
  Cisco’s Performance Routing (PfR) is an example of the augmented model which
  uses the Master Controller (MC) to push policy onto remote forwarding nodes.
  Another example includes offline path computation element (PCE) servers for
  automated MPLS TE tunnel creation. In both cases, a small set of routers
  (PfR border routers or TE tunnel head-ends) are modified, yet the remaining
  routers are untouched. This model has a lower impact on the existing network
  because the wholesale failure of the controller simply returns the network
  to the distributed model, which is a viable solution for many businses
  cases. The diagram that follows depicts the augmented SDN model.

  \addimg{sdn-augmented.jpg}{0.7}{SDN Model --- Augmented}

  \item \textbf{Hybrid:} This model is very similar to the augmented model
  except that controller-originated policy can be imposed anywhere in the
  network. This gives additional granularity to network administrators; the
  main benefit over the augmented model is that the hybrid model is always
  topology-independent. The topological restrictions of which nodes the
  controller can program/affect are not present in this model. Cisco’s
  Application Centric Infrastructure (ACI) is a good example of this model.
  ACI separates reachability from policy, which is critical from both
  survivability and scalability perspectives. This solution uses the
  Application Policy Infrastructure Controller (APIC) as the policy
  application tool. The failure of the centralized controller in these models
  has an identical effect to that of a controller in the augmented model; the
  network falls back to a distributed model. The impact of a failed controller
  is a more significant since more devices are affected by the controller’s
  policy when compared to the augmented model. The diagram that follows
  depicts the augmented SDN model.

  \addimg{sdn-hybrid.jpg}{0.7}{SDN Model --- Hybrid}

  \item \textbf{Centralized:} This is the model most commonly referenced when
  the phrase ``SDN'' is used. It relies on a single controller, which hosts
  the entire control-plane. Ultimately, this device commands all of the
  devices in the forwarding-plane. These controllers push their forwarding
  tables with the proper information (which doesn’t necessarily have to be an
  IP-based table, it could be anything) to the forwarding hardware as
  specified by the administrators. This offers very granular control, in many
  cases, of individual flows in the network. The hardware forwarders can be
  commoditized into white boxes (or branded white boxes, sometimes called
  brite boxes) which are often inexpensive. Another value proposition of
  centralizing the control-plane is that a ``device'' can be almost anything:
  router, switch, firewall, load-balancer, etc. Emulating software functions
  on generic hardware platforms can add flexibility to the business. \\
  
  The most significant drawback is the newly-introduced single point of
  failure and the inability to create failure domains as a result. Some SDN
  scaling architectures suggest simply adding additional controllers for fault
  tolerance or to create a hierarchy of controllers for larger networks. While
  this is a valid technique, it somewhat invalidates the ``centralized'' model
  because with multiple controllers, the distributed control-plane is reborn.
  The controllers still must synchronize their routing information using some
  network-based protocol and the possibility of inconsistencies between the
  controllers is real. When using this multi-controller architecture, the
  network designer must understand that there is, in fact, a distributed
  control-plane in the network; it has just been moved around. The failure of
  all controllers means the entire failure domain supported by those
  controllers will be inoperable. The failure of the communication paths
  between controllers could likewise cause inconsistent/intermittent problems
  with forwarding, just like a fully distributed control-plane. OpenFlow is a
  good example of a fully-centralized model. Nodes colored gray in the diagram
  that follows have no standalone control plane of their own, relying
  entirely on the controller.

  \addimg{sdn-centralized.jpg}{0.7}{SDN Model --- Centralized}
\end{enumerate}

These SDN designs warrant additional discussion, specifically around the
communications channels that allow them to function. An SDN controller sits
``in the middle'' of the SDN notional architecture. It uses \textbf{northbound}
and \textbf{southbound} communication paths to operate with other components
of the architecture. \\

The \textbf{northbound} interfaces are considered APIs which are interfaces to existing
business applications. This is generally used so that applications can make
requests of the network, which could include specific performance requirements
(bandwidth, latency, etc). Because the controller ``knows'' this information
by communicating with the infrastructure devices via management agents, it can
determine the best paths through the network to satisfy these constraints.
This is loosely analogous to the original intent of the Integrated Services
QoS model using Resource Reservation Protocol (RSVP) where applications would
reserve bandwidth on a per-flow basis. It is also similar to MPLS TE
constrained SPF (CSPF) where a single device can source-route traffic through
the network given a set of requirements. The logic is being extended to
applications with a controller ``shim'' in between, ultimately providing a
full network view for optimal routing. A REST API is an example of a
northbound interface. \\

The \textbf{southbound} interfaces include the control-plane protocol between the
centralized controller and the network forwarding hardware. These are the less
intelligent network devices used for forwarding only (assuming a centralized
model). A common control-plane used for this purpose would be OpenFlow; the
controller determines the forwarding tables per flow per network device,
programs this information, and then the devices obey it. Note that OpenFlow is
not synonymous with SDN\@; it is just an example of one southbound control-plane
protocol. Because the SDN controller is sandwiched between the northbound and
southbound interfaces, it can be considered ``middleware'' in a sense. The
controller is effectively able to evaluate application constraints and produce
forwarding-table outputs. \\

The image that follows depicts a very high-level diagram of the SDN layers as
it relates to interaction between components.

remake sdn ppt pic

There are many trade-offs between the different SDN models. The table that follows
attempts to capture the most important ones. Looking at the SDN market at the
time of this writing, many solutions seem to be either hybrid or augmented
models. SD-WAN solutions, such as Cisco Viptela, only make changes at the edge
of the network and use overlays/tunnels as the primary mechanism to implement policy.

\begin{longtable}{LLLLL}
\toprule
% top left cell is blank
&
\textbf{Distributed}
&
\textbf{Augmented}
&
\textbf{Hybrid}
&
\textbf{Centralized}
\\ \midrule
\textbf{Availability}
&
Dependent on the protocol convergence times and redundancy in the network.
Highly automonous and heals itself without a central brain
&
Dependent on the protocol convergence times and redundancy in the network.
Doesnt matter how bad the SDN controller is its failure is tolerable
&
Dependent on the protocol convergence times and redundancy in the network.
Doesnt matter how bad the SDN controller is  its failure is tolerable
&
Heavily reliant on a single SDN controller, unless one adds controllers to
split failure domains or to make one failure domain resilient
(both introduce a distributed control-plane)
\\ \midrule
\textbf{Granularity / control}
&
Generally low for IGPs but better for BGP\@. All devices generally need a common
view of the network to prevent loops independently. MPLS TE helps somewhat.
&
Better than distributed since policy injection can happen at the network edge,
or a small set of nodes. Can be combined with MPLS TE for more granular selection.
&
Moderately granular since SDN policy decisions are extended to all nodes. Can
influence decisions based on any arbitrary information within a datagram
&
Very highly granular; complete control over all routing decisions based on any
arbitrary information within a datagram
\\ \midrule
\textbf{Scalability}
&
Very high in a properly designed network (failure domain isolation, topology
summarization, reachability aggregation, etc)
&
High, but gets worse with more policy injection. Policies are generally
limited to key nodes (such as border routers)
&
Moderate, but gets worse with more policy injection. Policy is proliferated
across the network to all nodes (exact quantity may vary per node)
&
Depends; all devices retain state for all transiting flows. Hardware-dependent
on TCAM and whether it can use other tables such as L4 ports or IPv6 flow
labels
\\
\bottomrule
\end{longtable}
