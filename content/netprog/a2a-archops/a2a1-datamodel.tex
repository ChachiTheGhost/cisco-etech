\subsection{Data models and structures}
Other protocols and languages, such as NETCONF and YANG, also help
automate/simplify network management indirectly. NETCONF (RFC6241) is the
protocol by which configurations are installed and changed. YANG (RFC6020) is
the modeling language used to represent device configuration and state, much
like eXtensible Markup Language (XML). Put simply, NETCONF is a transport
vessel for YANG information to be transferred from a network management system
(NMS) to a network device. Although YANG can be quite complex to humans, it is
similar to SNMP; it is simple for machines. YANG is an abstraction away from
network device CLIs which promotes simplified management in cloud environments
and a progressive migration toward one of the SDN models discussed later in
this document. Devices that implement NETCONF/YANG provide a uniform
manageability interface which means vendor hardware/software can be swapped in
a network without affecting the management architecture, operations, or strategy.
\\ \\
Previous revisions of this document claimed that NETCONF is to YANG as HTTP is
to HTML. This is not technically accurate, as NETCONF serializes data using
XML. In HTML, Document Type Definitions (DTDs) describe the building blocks of
how the data is structured. This is the same role played by YANG as it relates
to XML. It would be more correct to say that DTD is to HTML over HTTP as YANG
is to XML over NETCONF. YANG can also be considered to be a successor to
Simple Network Management Protocol (SNMP) Management Information Base (MIB)
definitions. These MIBs define how data is structured and SNMP itself provides
a transport, similar to NETCONF.

\subsubsection{YANG}
YANG defines how data is structured/modeled rather than containing data
itself. Below is snippet from RFC 6020 which defines YANG (section 4.2.2.1).
The YANG model defines a "host-name" field as a string (array of characters)
with a human-readable description. Pairing YANG with NETCONF, the XML syntax
references the data field by its name to set a value.
\\ \\
YANG Example:

\begin{minted}{text}
leaf host-name {
    type string;
    description "Hostname for this system";
}
\end{minted}

NETCONF XML Example:
\begin{minted}{text}
<host-name>my.example.com</host-name>
\end{minted}

This section explores a YANG validation example using Cisco CSR1000v on modern
``Everest'' software. This router is running as an EC2 instance inside AWS.
Although the NETCONF router is not used until later, it is imported to check
the software version to ensure we clone the right YANG models.

\begin{minted}{text}
NETCONF_TEST#show version | include IOS_Software
Cisco IOS Software [Everest], Virtual XE Software (X86_64_LINUX_IOSD-UNIVERSALK9-M),
  Version 16.6.1, RELEASE SOFTWARE (fc2)
\end{minted}

YANG models for this particular version are publicly available on
\href{https://github.com/YangModels/yang/}{Github}. Below, the repository is
cloned using SSH which captures all of the YANG models for all supported
products, across all versions. The repository is not particularly large, so
cloning the entire thing is beneficial for future testing.

\begin{minted}{text}
Nicholas-MBP:YANG nicholasrusso$ git clone git@github.com:YangModels/yang.git
Cloning into 'yang'...
remote: Counting objects: 10372, done.
remote: Compressing objects: 100% (241/241), done.
remote: Total 10372 (delta 74), reused 292 (delta 69), pack-reused 10062
Receiving objects: 100% (10372/10372), 19.95 MiB | 4.81 MiB/s, done.
Resolving deltas: 100% (6556/6556), done.
Checking connectivity... done.
Checking out files: 100% (9212/9212), done.
\end{minted}

Changing into the directory specific to the current IOS-XE version, verify
that the EIGRP YANG model used for this test is present. There are 245 total
files in this directory which are the YANG models for many other Cisco
features, but are outside the scope of this demonstration.

\begin{minted}{text}
Nicholass-MBP:~ nicholasrusso$ cd yang/vendor/cisco/xe/1661/
Nicholass-MBP:1661 nicholasrusso$ ls -1 | grep eigrp
Cisco-IOS-XE-eigrp.yang
Nicholass-MBP:1661 nicholasrusso$ ls -1 | wc
     245    2198   20770
\end{minted}

The YANG model itself is a C-style declaration of how data should be
structured. The file is very long, and the text below focuses on a few key
EIGRP parameters. Specifically, the bandwidth-percent, hello-interval, and
hold-time. These are configured under the af-interface stanza within EIGRP
named-mode. The af-interface declaration is a list element with many leaf
elements beneath it, which correspond to individual configuration parameters.

\begin{minted}{text}
Nicholass-MBP:1661 nicholasrusso$ cat Cisco-IOS-XE-eigrp.yang
module Cisco-IOS-XE-eigrp {
  namespace "http://cisco.com/ns/yang/Cisco-IOS-XE-eigrp";
  prefix ios-eigrp;

  import ietf-inet-types {
    prefix inet;
  }

  import Cisco-IOS-XE-types {
    prefix ios-types;
  }

  import Cisco-IOS-XE-interface-common {
    prefix ios-ifc;
  }
[snip]
 // router eigrp address-family
  grouping eigrp-address-family-grouping {
    list af-interface {
      description
        "Enter Address Family interface configuration";
      key "name";
      leaf name {
        type string;
      }
      leaf bandwidth-percent {
        description
          "Set percentage of bandwidth percentage limit";
        type uint32 {
          range "1..999999";
        }
      }
      leaf hello-interval {
        description
          "Configures hello interval";
        type uint16;
      }
      leaf hold-time {
        description
          "Configures hold time";
        type uint16;
}
[snip]
\end{minted}

Before exploring NETCONF, which will use this model to get/set configuration
data on the router, this demonstration explores the \verb|pyang| tool. This is a
conversion tool to change YANG into different formats. The pyang tool is
available \href{https://pypi.python.org/pypi/pyang}{here}. After extracting the
archive, the tool is easily installed.

\begin{minted}{text}
Nicholass-MBP:pyang-1.7.3 nicholasrusso$ python3 setup.py install
running install
running bdist_egg
running egg_info
writing top-level names to pyang.egg-info/top_level.txt
writing pyang.egg-info/PKG-INFO
writing dependency_links to pyang.egg-info/dependency_links.txt
[snip]

Nicholass-MBP:pyang-1.7.3 nicholasrusso$ which pyang
/Library/Frameworks/Python.framework/Versions/3.5/bin/pyang
\end{minted}

The most basic usage of the pyang tool is to validate valid YANG syntax.
Beware that running the tool against a YANG model in a different directory
means that pyang considers the local directory (not the one containing the
YANG model) for the search point for any YANG module dependencies. Below, an
error occurs since pyang cannot find imported modules relevant for the EIGRP
YANG model.

\begin{minted}{text}
Nicholass-MBP:YANG nicholasrusso$ pyang yang/vendor/cisco/xe/1661/Cisco-IOS-XE-eigrp.yang 
yang/vendor/cisco/xe/1661/Cisco-IOS-XE-eigrp.yang:5: error: module "ietf-inet-types" not found in search path
yang/vendor/cisco/xe/1661/Cisco-IOS-XE-eigrp.yang:10: error: module "Cisco-IOS-XE-types" not found in search path
[snip]

Nicholass-MBP:YANG nicholasrusso$ echo $?
1
\end{minted}

One could specify the module path using the \verb|--path option|, but it is
simpler to just navigate to the directory. This allows pyang to see the
imported data types such as those contained within ``ietf-inet-types''. When
using pyang from this location, no output is returned, and the program exits
successfully. It is usually a good idea to validate YANG models before doing
anything with them, especially committing them to a repository.

\begin{minted}{text}
Nicholass-MBP:YANG nicholasrusso$ cd yang/vendor/cisco/xe/1661/
Nicholass-MBP:1661 nicholasrusso$ pyang Cisco-IOS-XE-eigrp.yang
Nicholass-MBP:1661 nicholasrusso$ echo $?
0
\end{minted}

This confirms that the model has valid syntax. The pyang tool can also convert
between different formats. Below is a simple and lossless conversion of YANG
syntax into XML. This YANG-to-XML format is known as YIN, and pyang can
generate pretty XML output based on the YANG model. This is an alternative way
to view, edit, or create data models. YIN format might be useful for Microsoft
Powershell users. Powershell makes XML parsing easy, and may not be as
friendly to the YANG syntax.

\begin{minted}{text}
Nicholass-MBP:1661 nicholasrusso$ pyang Cisco-IOS-XE-eigrp.yang \
>  --format=yin --yin-pretty-strings
<?xml version="1.0" encoding="UTF-8"?>
<module name="Cisco-IOS-XE-eigrp"
     xmlns="urn:ietf:params:xml:ns:yang:yin:1"
     xmlns:ios-eigrp="http://cisco.com/ns/yang/Cisco-IOS-XE-eigrp"
     xmlns:inet="urn:ietf:params:xml:ns:yang:ietf-inet-types"
     xmlns:ios-types="http://cisco.com/ns/yang/Cisco-IOS-XE-types"
     xmlns:ios-ifc="http://cisco.com/ns/yang/Cisco-IOS-XE-interface-common"
     xmlns:ios="http://cisco.com/ns/yang/Cisco-IOS-XE-native">
  <namespace uri="http://cisco.com/ns/yang/Cisco-IOS-XE-eigrp"/>
  <prefix value="ios-eigrp"/>
  <import module="ietf-inet-types">
    <prefix value="inet"/>
  </import>
  <import module="Cisco-IOS-XE-types">
    <prefix value="ios-types"/>
  </import>
  <import module="Cisco-IOS-XE-interface-common">
    <prefix value="ios-ifc"/>
  </import>
[snip]
  <grouping name="eigrp-address-family-grouping">
    <list name="af-interface">
      <description>
        <text>
          Enter Address Family interface configuration
        </text>
      </description>
      <key value="name"/>
      <leaf name="name">
        <type name="string"/>
      </leaf>
      <leaf name="bandwidth-percent">
        <description>
          <text>
            Set percentage of bandwidth percentage limit
          </text>
        </description>
        <type name="uint32">
          <range value="1..999999"/>
        </type>
      </leaf>
      <leaf name="hello-interval">
        <description>
          <text>
            Configures hello interval
          </text>
        </description>
        <type name="uint16"/>
      </leaf>
      <leaf name="hold-time">
        <description>
          <text>
            Configures hold time
          </text>
        </description>
        <type name="uint16"/>
      </leaf>   
[snip]
\end{minted}

Everything shown above is unrelated to the NETCONF/YANG testing on the Cisco
CSR1000v and is more focused on viewing, validating, and converting YANG
models between different formats. These YANG models are already loaded into
the Cisco IOS-XE images and do not need to be present on the management
station's disks. Please see the NETCONF section for more information.

\subsubsection{YAML}
There are many options for storing data, as opposed to modeling it or defining
its composition. One such option is YAML Ain’t Markup Language (YAML). It
solves a similar problem as XML since it is primarily used for configuration
files, but contains a subset of XML's functionality as it was specifically
designed to be simpler. Below is an example of a YAML configuration, most
likely some input for a provisioning script or something similar. Note that
YAML files typically begin with \verb|---| and end with \verb|...| as a best practice.

\begin{minted}{yaml}
---
- process: "update static routes"
  vrf: "customer1"
  nexthop: "10.0.0.1"
  devices:
    - net: "192.168.0.0"
      mask: "255.255.0.0"
      state: "present"
    - net: "172.16.0.0"
      mask: "255.240.0.0"
      state: "absent"
...
\end{minted}

Note that YAML is technically not in the blueprint but the author is certain
it is a critical skill for anyone working with any form of network programmability.

\subsubsection{JSON}
JavaScript Object Notation (JSON) is another data modeling language that is
similar to YAML in concept. It was designed to be simpler than traditional
markup languages and uses key/value pairs to store information. The "value" of
a given pair can be another key/value pair, which enables hierarchical data
nesting. The key/value pair structure and syntax is very similar to the
"dictionary" data type in Python. JSON is even lighter than YAML and is also
commonly used for maintaining configuration files. The next page displays a
syntax example of JSON which represents the same data and same structure as
the YAML example.

\begin{minted}{json}
[
  {
    "process": "update static routes",
    "vrf": "customer1",
    "nexthop": "10.0.0.1",
    "devices": [
      {
        "net": "192.168.0.0",
        "mask": "255.255.0.0",
        "state": "present"
      },
      {
        "net": "172.16.0.0",
        "mask": "255.240.0.0",
        "state": "absent"
      }
    ]
  }
]
\end{minted}

More discussion around YAML and JSON is warranted since these two formats are
very commonly used today. YAML is considered to be a strict (or proper)
superset of JSON. That is, any JSON data can be represented in YAML, but not
vice versa. This is important to know when converting back and forth;
converting JSON to YAML should always succeed, but converting YAML to JSON may
fail or yield unintended results. Below is a straightforward conversion
between YAML and JSON without any hidden surprises.

\begin{minted}{yaml}
---
mylist:
  - item: "pizza"
    quantity: 1
  - item: "wings"
    quantity: 12
...
\end{minted}

\begin{minted}{json}
{
  "mylist": [
    {
      "item": "pizza",
      "quantity": 1
    },
    {
      "item": "wings",
      "quantity": 12
    }
  ]
}
\end{minted}

Next, observe an example of some potentially unexpected conversion results.
While the JSON result is technically correct, it lacks the shorthand
``anchoring'' technique available in YAML. The anchor, for example, creates
information that can be inherited by other dictionaries later. While the
information is identical between these two blocks and has no functional
difference, some of these YAML shortcuts are advantageous for encouraging
code/data reuse. Another difference is that YAML natively supports comments
using the hash symbol \verb|#| while JSON does not natively support comments.

\begin{minted}{yaml}
---
anchor: anchor&
  name: "Nick"
  age: 31

clone:
  <<: *anchor
...
\end{minted}

\begin{minted}{json}
{
  "anchor": {
    "name": "Nick",
    "age": 31
  },
  "clone": {
    "name": "Nick",
    "age": 31
  }
}
\end{minted}

YANG isn’t directly comparable with YAML, JSON, and XML because it solves a
different problem. If any one of these languages solved all of the problems,
then the others would not exist. Understanding the business drivers and the
problems to be solved using these tools is the key to choosing to right one.

\subsubsection{XML}
Data structured in XML is very common and has been popular for decades. XML is
very verbose and explicit, relying on starting and ending tags to identify the
size/scope of specific data fields. The next page shows an example of XML code
resembling a similar structure as the previous YAML and JSON examples. Note
that the topmost ``root'' wrapper key is needed in XML but not for YAML or JSON.

\begin{minted}{xml}
<?xml version="1.0" encoding="UTF-8" ?>
	<root>
		<process>update static routes</process>
		<vrf>customer1</vrf>
		<nexthop>10.0.0.1</nexthop>
		<devices>
			<net>192.168.0.0</net>
			<mask>255.255.0.0</mask>
			<state>present</state>
		</devices>
		<devices>
			<net>172.16.0.0</net>
			<mask>255.240.0.0</mask>
			<state>absent</state>
		</devices>
	</root>
\end{minted}
